
\section{Preliminaries\label{sec:proof_outline}}

In this section we recall some definitions and tools that we will need in our proofs.

\subsection{The infinite limit model $\Ginf$\label{ssec:infinite_model}}

We start by recalling the definition of the infinite limit model from~\cite{fountoulakis2018law}.
Let $\mathcal{P}=\mathcal{P}_{\alpha,\nu}$ 
be a Poisson point process on $\eR^2$ with intensity function $f=f_{\alpha,\nu}$ given by

\begin{equation}\label{eq:def_intensity_function_f}
	f(x,y) = \frac{\alpha \nu}{\pi} e^{-\alpha y} \cdot 1_{\{y>0\}}.
\end{equation} 

The \emph{infinite limit model} $\Ginf = \Ginf(\alpha,\nu)$ has vertex set $\mathcal{P}$ and edge set such that
\[
	pp' \in E(\Ginf) \iff |x - x'| \leq e^{\frac{y + y'}{2}},
\]
for $p=(x,y),p'=(x',y')\in\Pcal$.

For any point $p \in \Rcal$, we write $\BallPo{p}$ to denote the \emph{ball} around $p$, i.e.

\begin{equation}\label{eq:def_ball_P}
	\BallPo{p} = \{p^\prime \in \eR\times[0,\infty) : |x - x^\prime| \leq e^{\frac{y + y^\prime}{2}}\}.
\end{equation}

With this notation we then have that $\BallPo{p} \cap \Pcal$ denotes the set of neighbors of a vertex $p \in \Ginf$.
We will denote the intensity measure of the Poisson process $\mathcal{P}$ by $\mu = \mu_{\alpha, \nu}$, i.e. for every 
Borel-measurable subset $S \subseteq \eR^2$ we have $\mu(S) = \int_S f(x,y) \, dx \, dy$


\subsection{The finite box model $\Gbox$\label{ssec:finite_model}}


For the definition of the finite graph, recall that in the definition of the KPKVB model we set $R = 2\log(n/\nu)$.
We consider the box $\Rcal = (-\frac{\pi}{2}e^{R/2}, \frac{\pi}{2}e^{R/2}] \times (0, R]$ in $\eR^2$. 
Then the \emph{finite box model} $\Gbox := \Gbox(n;\alpha, \nu)$ has vertex set $\mathcal{V} := \mathcal{P} \cap \Rcal$ and edges set such that
\[
	pp' \in E(\Gbox(n;\alpha, \nu)) \iff |x - x'|_{\pi e^{R/2}} \leq e^{\frac{y + y'}{2}},
\]
where $|x|_{r} = \min( |x|, r - |x|)$ for $-r\leq x\leq r$. 
Using $|.|_{\pi e^{R/2}}$ instead of $|.|$ results in the left and right boundaries of the box $\Rcal$ getting 
identified, which in particular 
%excludes boundary effects and 
makes the model invariant under horizontal shifts and reflections in vertical lines. 
The graph $\Gbox$ can thus be seen as a subgraph of $\Ginf$ induced on $\mathcal{V}_n$, with some additional edges caused by 
the identification of the boundaries.

Similar to the infinite graph, for a point $p \in \Rcal$ we define the ball $\BallPon{p}$ as

\begin{equation}\label{eq:def_ball_P_n}
	\BallPon{p} = 
	\left\{p' \in \Rcal : |x - x'|_{\pi e^{R/2}} \leq e^{\frac{y + y'}{2}}\right\}.
\end{equation}


\subsection{The Poissonized KPKVB model $\GPo$}


Imagine that we have an infinite supply of i.i.d.~points $p_1, p_2, \dots$ in the hyperbolic plane $\Haa$ chosen according to the $(\alpha, R)$-quasi uniform distribution. In the standard KPKVB random graph $G(n;\alpha,\nu)$ we take $p_1,\dots, p_n$ as our vertex set and add edges between points at hyperbolic distance at most $R = 2\log(n/\nu)$. In the {\em Poissonized} KPKVB random graph $\GPo := \GPo(n;\alpha,\nu)$, we instead take $N\isd \Po(n)$, a Poisson random variable 
with mean $n$, independent of our i.i.d.~sequence of points and let the vertex set be $p_1,\dots, p_N$ and add edges 
according to the same rule as before. Equivalently, we could say that the vertex set consists of the points of a Poisson point process with intensity function $n g$, where $g$ denotes the probability density of the $(\alpha,R)$-quasi uniform distribution. That is,
\[
	g(r,\theta) = \frac{\alpha\sinh(\alpha r)}{2\pi(\cosh(\alpha R) - 1)} \cdot 1_{\{0\leq r\leq R, -\pi<\theta\leq \pi\}}.
\] 

Working with the Poissonized model has the advantage that when we take two disjoint regions $A, B$ then the number of points in $A$ and the number of points  in $B$ are independent Poisson-distributed random variables. As we will see, and as is to be expected, switching to the Poissonized model does not significantly alter the limiting behaviour of the clustering coefficient and function.


\subsection{Coupling $\GPo$ and $\Gbox$\label{ssec:coupling_H_P}}


The following lemmas from \cite{fountoulakis2018law} establish a useful coupling between the Poissonized KPKVB random graph
and the finite box model and relate the edge  sets of the two graphs. 

\begin{lemma}[{\cite[Lemma 27]{fountoulakis2018law}}]\label{lem:coupling_hyperbolic_poisson}
Let $\mathcal{V}$ denote the vertex set of $\GPo(n;\alpha, \nu)$ and $\mathcal{W}$ the vertex set of $\Gbox(n;\alpha, \nu)$. 
Define the map $\Psi : [0,R] \times (-\pi, \pi] \to \Rcal$ by
\begin{equation}\label{eq:def_Psi}
	\Psi(r,\theta) = \left(\theta \frac{e^{R/2}}{2}, R - r\right).
\end{equation}
Then there exists a coupling such that, a.a.s., $\mathcal{W} = \Psi[\mathcal{V}]$. %Moreover, if $\Ccal_n$ is the event that ${\cal V}_n = \Psi\left(\mathcal{V}_{\HP,n}\right)$ then
%\begin{equation}\label{eq:convergence_miscoupling_hyperbolic_poisson}
%	\Prob{\Ccal_n^c} = \bigO{n^{-(2\alpha - 1)}}. 
%\end{equation}
\end{lemma}

In the remainder of this paper we will write $\BallHyp{p}$ to denote the image under $\Psi$ of the ball of hyperbolic radius $R$ around the point 
$\Psi^{-1}(p)$, i.e. 
\[
	\BallHyp{p} := 
	%\left\{p^\prime := \Psi(u) \, : \, u \in \Dcal_{R_n} \text{ and } d_\H(\Psi^{-1}(p),u) \le R_n\right\}.
	\Psi\left[ \left\{ u \in \Haa : 
	d_\H(\Psi^{-1}(p),u), d_\H(O,u) \le R \right\}\right].
\]

Under the map $\Psi$, a point $p = (x,y) \in \Rcal$ corresponds to $u := \Psi^{-1}(p) = (2 e^{-R/2} x, R - y)$. 

By the hyperbolic rule of cosines, for two points $p = (x,y) = \Psi( (r,\theta) ), p' = (x',y') = \Psi( (r',\theta') ) \in \Rcal$ we have that
$p' \in \BallHyp{p}$ iff.~either $r+r'\leq R$ or $r+r'>R$ and

$$
	\cosh r \cosh r' - \sinh r \sinh r'\cos\left( |\theta-\theta'|_{2\pi} \right) \le \cosh(R),
$$

\noindent
This can be rephrased as $p'\in \BallHyp{p}$ iff.~either $y+y'\geq R$ or $y+y'<R$ and 


\begin{equation}\label{eq:def_Omega_hyperbolic}
	|x-x'|_{\pi e^{R/2}} \leq \Phi(r,r') := \frac{1}{2}e^{R/2} \arccos\left( \frac{\cosh r \cosh r' - \cosh R_n}
	{\sinh r \sinh r'} \right).
\end{equation}

The following lemma provides useful bounds on the function $\Phi(r,r^\prime)$.

\begin{lemma}[{\cite[Lemma 28]{fountoulakis2018law}}]\label{lem:asymptotics_Omega_hyperbolic}
There exists a constant $K>0$ such that, for every $\varepsilon > 0$ and for $R$ sufficiently large, the following holds.
For every $r,r^\prime \in [\varepsilon R,R]$ with $r + r^\prime > R$ we have that 
\begin{equation}\label{eq:asymp1}
	e^{\frac{1}{2}(y+y^\prime)} - K e^{\frac{3}{2}(y+y^\prime) - R} \leq \Phi(r, r^\prime) 
	\leq  e^{\frac{1}{2}(y+y^\prime)} + K e^{\frac{3}{2}(y+y^\prime) - R},
\end{equation}
where $y := R - r, y^\prime := R - r^\prime$. 
Moreover:
\begin{equation}\label{eq:asymp2} 
\Phi(r,r^\prime) \geq e^{\frac12(y+y^\prime)} \quad \text{if \quad $r, r^\prime < R - K$.} 
\end{equation}
\end{lemma}

A key consequence of Lemma~\ref{lem:asymptotics_Omega_hyperbolic} is that the coupling from Lemma~\ref{lem:coupling_hyperbolic_poisson} 
preserves edges between points whose heights are not too large.  

\begin{lemma}[{\cite[Lemma 30]{fountoulakis2018law}}]\label{lem:coupling_edges}
On the coupling space of Lemma~\ref{lem:coupling_hyperbolic_poisson} the following holds a.a.s.:
\begin{enumerate}
\item for any two points $p, p' \in \mathcal{W}$ with $y, y'\le R/2$, we have 

$$pp' \in E(\Gbox) \Rightarrow \Psi^{-1}(p)\Psi^{-1}(p') \in E(\GPo), $$ 

\item for any two points $p, p' \in \mathcal{W}$ with $y, y' \le R/4$, we have that 

$$pp' \in E(\Gbox) \iff \Psi^{-1}(p)\Psi^{-1}(p') \in E(\GPo).$$

\end{enumerate}
\end{lemma}


\subsection{The Campbell-Mecke formula}

\PvdH{We might want to give some more results for Poisson point processes and Poisson random variables. Such as at the end of Section 2.1 in your recent paper on Hamiltonian cycles.}

A very useful tool for analyzing subgraph counts, and their generalizations, in the 
setting of the Poissonized random geometric graphs, and in particular the Poissonized KPKVB model and the box model is the \emph{Campbell-Mecke formula}.
We use the incarnation below, which can be found in the monograph~\cite{penrose2003random}, as Theorem 1.6.

\PvdH{I found it extremely hard to see the implications of this theorem. I therefore strongly recommend using more explicit version, for instance as in the previous version of the paper, or explicitly state the consequences of this result which will be used here.}

\begin{theorem}[\cite{penrose2003random}]\label{thm:palm}
Let $\Qcal$ be a Poisson process on $\eR^d$ with intensity function $g$, and suppose that $\lambda := \int g < \infty$.
Suppose that $h(\Ycal, \Xcal)$ is a bounded measurable function, defined on pairs $(\Ycal,\Xcal)$ with $\Ycal \subseteq \Xcal \subseteq \eR^d$
and $\Xcal$ finite, such that $h(\Ycal,\Xcal) = 0$ whenever $|\Ycal| \neq j$ (for some $j \in \eN$).
Then
\[ \Ee \sum_{\Ycal \subseteq \Qcal} h( \Ycal, \Qcal) = \frac{\lambda^j}{j!} \cdot \Ee h(\{Y_1,\dots, Y_j\}, \{Y_1,\dots, Y_j\} \cup \Qcal ), \]
\noindent
where the $Y_i$ are i.i.d.~random variables that are independent of $\Pcal$ and have common probability density function
$g/\lambda$.
\end{theorem}

% Let $\mathcal{Q}$ be a point process on a metric space $S$ with density $\rho$. 
% Let $\mathcal{N} = \mathcal{N} (\mathcal{S})$ be the set of all countable point configurations in $\mathcal{S}$
% equipped with the $\sigma$-algebra of the point process (that is, for any open subset 
% $A\subseteq S$ and any non-negative integer $m$ define a basic measurable subset of 
% $\mathcal{N}$ which consists of all configurations which have exactly $m$ points in $A$). 
% Now, let $h : S^{k} \times \mathcal{N} \rightarrow \eR$ be a measurable function. 
% The Palm theory of Poisson point processes on metric spaces~\cite{bk:LastPenrose} yields: 
% 
% \begin{equation} \label{eq:Mecke}
% \E \left( \sum_{p_1, \ldots, p_k \in {\mathcal Q}, \atop \text{distinct}} h(p_1,\ldots, p_k, \mathcal{Q})  \right) =
% \int_S \cdots \int_S \E (h(x_1,\ldots, x_k, \mathcal{Q})) d\rho (x_1) \cdots d \rho (x_k),
% \end{equation} 
% 
% where the sum ranges over all those $k$-tuples of points which contain no repetitions. 
% Note that in the RHS, the points $x_1, \dots, x_k$ are not part of $\Qcal$.

