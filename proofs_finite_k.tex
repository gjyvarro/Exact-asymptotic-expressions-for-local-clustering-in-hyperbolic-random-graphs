


\section{Proofs of Theorem~\ref{thm:maincc} and Theorem~\ref{thm:mainkfixed}\label{sec:proofs_fixed_k}}


We will first derive Theorem~\ref{thm:mainkfixed}. It will turn out that Theorem~\ref{thm:maincc} has a quick derivation
assuming Theorem~\ref{thm:mainkfixed}.

\subsection{Clustering function, proving Theorem~\ref{thm:mainkfixed}}
In this subsection, we want to show that the clustering function of the KPKVB model $c_n(k) \xrightarrow{\Pee} \gamma(k)$ for a fixed $k$. The key idea is that for the poissonized KPKVB model the coupling with the box model is guaranteed to be exact (in the sense that it also preserves edges) for all vertices up to height $H = \frac{R}{4}$ (which is all we need to show the convergence of the clustering function for constant $k$).

The overall proof proceeds in four steps: firstly poissonizing the KPKVB model, secondly truncating the poissonized KPKVB model at height $H=H_n \rightarrow \infty$, thirdly using that the coupling between the poissonized KPKVB and box model is exact when truncated at this height $H$ and finally showing that the clustering function in the truncated box model agrees with the integral for $\gamma(k)$.

We will encouter the following models: we will start with the KPKVB model $G_n = G(n;\alpha,\nu)$, then the Poissonized KPKVB model $\GPo  = G_{Po}(n;\alpha,\nu)$, moving to the truncated Poissonized KPKVB model $\GPoH$ which is the subgraph of $\GPo$ induced on all vertices with height at most $H$ resp. radial coordinate at least $R-H$, then the coupling will bring us to the truncated box model $\GboxH$ which is the subgraph of the box model $G_{box}$ induced by all vertices with height at most $H$.

\subsubsection{Poissonization}
We first show we can restrict attention to the Poissonized instead of the standard KPKVB model.

\begin{lemma}\label{lem:pok} Let $\alpha>\frac12,\nu>0, k\geq 2$ be fixed. % and write $G_n := G(n;\alpha,\nu), G_{n,\text{Po}} := \GPo(n;\alpha,\nu)$.
Then 
$$ c(k; G_n)-c(k; \GPo) \xrightarrow[n\to\infty]{\Pee} 0. $$
\end{lemma}

\begin{proof}
We consider the coupling where we have an infinite supply of i.i.d.~points $u_1, u_2, \dots$ chosen according
to the $(\alpha,R)$-quasi uniform distribution, the vertices of $G(n;\alpha,\nu)$ are $u_1,\dots, u_n$ and 
the vertices of $\GPo(n;\alpha,\nu)$ are $u_1,\dots, u_N$ with $N\isd \Po(n)$ independently of
$u_1, u_2, \dots$.

Due to the coupling, the KPKVB graph or its poissonization is contained in the other, i.e. we can use new graph labels $G_1 \subset G_2$; e.g. if $N\geq n$, then $G_1 = G(n;\alpha,\nu)$ and $G_2 = \GPo(n;\alpha,\nu)$. We write $V_{1,k}$ for the set of vertices of $G_1$ which have $k$ neighbours in $G_1$ and $V_{2,k}$ for the set of vertices of $G_2$ which have $k$ neighbours in $G_2$. Write $N_{1,k}=|V_{1,k}|$ and $N_{2,k}=|V_{2,k}|$.

Write $W_1$ for the set of all vertices of $G_1$ which have $k$ neighbours in $G_1$ but no neighbour in $G_2 \backslash G_1$. Write $W_2$ for the set of all vertices of $G_1$ which have $k$ neighbours in $G_1$ and at least one neighbour in $G_2 \backslash G_1$. Write $W_3$ for the set of all vertices of $G_2$ which have strictly less than $k$ neighbours in $G_1$, but $k$ neighbours in $G_2$.

With this notation, we can rewrite:
\begin{align*}
|c(k;G_n)-c(k;\GPo)| &= \left|\sum_{v \in V_{1,k}}\frac{c_{G_1}(v)}{N_{1,k}}-\sum_{v \in V_{2,k}}\frac{c_{G_2}(v)}{N_{2,k}}\right| \\
&=\left|\sum_{v \in W_1}\frac{c_{G_1}(v)}{N_{1,k}}+\sum_{v \in W_2}\frac{c_{G_1}(v)}{N_{1,k}}-\sum_{v \in W_1}\frac{c_{G_2}(v)}{N_{2,k}}-\sum_{v \in W_3}\frac{c_{G_2}(v)}{N_{2,k}}\right|
\end{align*}

For every vertex of $W_1$, the clustering coefficient in $G_1$ and in $G_2$ have the same value. Hence, we can simplify:
\begin{align*}
|c(k;G_n)-c(k;\GPo)| &= \left|\sum_{v \in W_1}c_{G_1}(v)\frac{N_{2,k}-N_{1,k}}{N_{1,k}N_{2,k}}+\sum_{v \in W_2}\frac{c_{G_1}(v)}{N_{1,k}}-\sum_{v \in W_3}\frac{c_{G_2}(v)}{N_{2,k}}\right|
\end{align*}
Using the triangle inequality for the absolute value and the fact that $|c_G(v)|\leq 1$,
\begin{align*}
|c(k;G_n)-c(k;\GPo)| &\leq \frac{|W_1| |N_{2,k}-N_{1,k}|}{N_{1,k}N_{2,k}}+\frac{|W_2|}{N_{1,k}}+\frac{|W_3|}{N_{2,k}}
\end{align*}
Using that $|W_1|\leq N_{1,k}$,
\begin{align*}
|c(k;G_n)-c(k;\GPo)| &\leq \frac{|N_{2,k}-N_{1,k}|}{N_{2,k}}+\frac{|W_2|}{N_{1,k}}+\frac{|W_3|}{N_{2,k}}
\end{align*}

From the Chernoff bound for Poisson random variables, it follows that for any $c>0$, $N \in [n-c\sqrt{n\log n},n+c\sqrt{n\log n}]$ a.a.s.. This implies that a.a.s. the vertex sets of $G_1$ and $G_2$ differ by at most $c\sqrt{n\log n}$ many vertices. In particular also $|N_{1,k}-N_{2,k}|,|W_2|,|W_3|=O(\sqrt{n \log n})$ a.a.s. 

By the paper by Gugelmann et al.~\cite{gugelmann2012random} it holds that $N_{1,k}$ or $N_{2,k}$ is $\Theta(n)$ a.a.s. for fixed degree $k$ and as $|N_{1,k}-N_{2,k}| =O(\sqrt{n\log n})$ a.a.s., it follows that both $N_{1,k},N_{2,k} = \Theta(n)$ a.a.s. Putting everything together yields the claim:
\begin{align*}
|c(k;G_n)-c(k;\GPo)| &\leq \frac{|N_{2,k}-N_{1,k}|}{N_{2,k}}+\frac{|W_2|}{N_{1,k}}+\frac{|W_3|}{N_{2,k}} \xrightarrow[n\rightarrow\infty]{\Pee} 0
\end{align*}


\end{proof}


\subsubsection{Truncation}
Consider the truncated Poissonized KPKVB model $\GPoH$ with $H =H_n= \frac{R}{4}$.
%H,R-H tend to infinity and coupling is still exact
\begin{lemma}\label{lem:n1}
Let $N_1(H)$ denote the number of vertices with radial coordinate at most $R-H \in [0,R]$ in the Poissonized KPKVB model $\GPo$. If $H \rightarrow \infty$, then $\E N_1(H) = o(n)$.
%For all $\epsilon >0$, there is $H_\epsilon$ such that for all $H\geq H_\epsilon$, it holds that $\E N_1(H) \leq \epsilon n$.
\end{lemma}
\begin{proof}
By the Campbell-Mecke formula applied to the vertex set of $\GPo$, by the asymptotic behaviour of $\cosh$ and using that all $R, R-H, H \rightarrow \infty$,
\begin{align*}
\E N_1(H) = n \frac{\cosh(\alpha(R-H))-1}{\cosh(\alpha R) -1} = (1+o(1)) n e^{-\alpha H} = o(n)
\end{align*}
\end{proof}
\begin{lemma}\label{lem:n2}
Let $N_2(H)$ denote the number of vertices with at least one neighbour with radial coordinate at most $R-H \in [0,R]$ in the Poissonized KPKVB model. If $H \rightarrow \infty$, then $\E N_2(H) = o(n)$.
\end{lemma}
\begin{proof}
Among the vertices with at least one neighbour with radial coordinate at most $R-H$, we distinguish between those that themselves have radial coordinate at most $R-H_0$ and those that themselves have radial coordinate at least $R-H_0$ where $H_0 = \log H$. We will show that the expectation of both numbers is sublinear in $n$.

If $H \rightarrow \infty$, then also $H_0 = \log H \rightarrow \infty$. Hence by Lemma~\ref{lem:n1}, the expected number of vertices with radial coordinate at most $R-H_0$ (in particular those vertices among them with at least one neighbour with radial coordinate at most $R-H$), $\E N_1(H_0) = o(n)$ is sublinear. For a vertex with radial coordinate at least $R-H_0$, the probability that it has a neighbour with radial coordinate at most $R-H$, tends to zero. Therefore, by the Campbell-Mecke formula the number of such vertices is sublinear.
\end{proof}

\begin{lemma}\label{lem:trunc}
Let $\alpha>\frac{1}{2}$, $\nu>0$ and $k \geq 2$ be fixed. Let $\GPo$ denote the Poissonized KPKVB model and $\GPoH$ denote the truncated Poissonized KPKVB model. Then 
$$c(k;\GPo)-c(k;\GPoH) \xrightarrow[n \rightarrow \infty]{\Pee} 0.$$
\end{lemma}
\begin{proof}
Let $V_n(k)$ denote the set of all vertices with degree $k$ in the Poissonized KPKVB model $\GPo$ and $V_{n,H}(k)$ the set of all vertices with degree $k$ in the truncated Poissonized KPKVB model $\GPoH$. Note that a vertex may have a smaller degree and different clustering coefficient in $\GPo$ and $\GPoH$ (even if it is a vertex of both graphs). We write $c_n(v)$ for the clustering coefficient of a vertex $v$ in $\GPo$ and $c_{n,H}(v)$ for the clustering coefficient in $\GPoH$. Let $N_n(k)=|V_n(k)|$ denote the number of degree $k$ vertices in $\GPo$. Then as $N_n(k) = \Theta(n k^{-(2\alpha+1)})$ a.a.s., there is a constant $a_1(k)>0$ such that a.a.s. $N_n(k) \geq a_1(k) n$. Let $N_{n,H}(k) = |V_{n,H}(k)|$ be the number of degree $k$ vertices in $\GPoH$.

Using a telescoping sum, we obtain that
\begin{align*}
|c(k;\GPo)-c(k;\GPoH)| \leq & \left|c(k;\GPo)-\frac{1}{N_n(k)}\sum_{v \in V_{n,H}(k)} c_n(v)\right|\\
&+\left|\frac{1}{N_n(k)}\sum_{v \in V_{n,H}(k)} c_n(v) -\frac{1}{N_n(k)}\sum_{v \in V_{n,H}(k)}c_{n,H}(v)\right| \\
&+\left|\frac{1}{N_n(k)}\sum_{v \in V_{n,H}(k)} c_{n, H}(v)-c(k;\GPoH)\right| 
\end{align*}
Now we use $X_1 := |V_n(k)\backslash V_{n,H}(k)|$, $X_2 := \sum_{v \in V_{n,H}(k)}|c_n(k)-c_{n,H}(k)|$ and $X_3 :=|N_{n,H}(k)-N_n(k)|$ to bound the three differences. In the first difference, after factorizing $N_n(k)$, all terms in $V_{n,H}(k)$ cancel and the remaining terms are all bounded by one:
\begin{align*}
\left|c(k;\GPo)-\frac{1}{N_n(k)}\sum_{v \in V_{n,H}(k)} c_n(v)\right| = \left| \frac{1}{N_n(k)}\sum_{v \in V_n \backslash V_{n,H}(k)} c_n(v) \right| \leq \frac{X_1}{N_n(k)}
\end{align*}
In the second difference, the sums can be aligned and then the triangle inequality for the absolute value applied:
\begin{align*}
\left|\frac{1}{N_n(k)}\sum_{v \in V_{n,H}(k)} c_n(v) -\frac{1}{N_n(k)}\sum_{v \in V_{n,H}(k)}c_{n,H}(v)\right| \leq \frac{1}{N_n(k)}\sum_{v \in V_{n,H}(k)} \left|c_n(v)-c_{n,H}(v)\right| =\frac{X_2}{N_n(k)}
\end{align*}
In the third difference, we can simplify:
\begin{align*}
\left|\frac{1}{N_n(k)}\sum_{v \in V_{n,H}(k)} c_{n, H}(v)-c(k;\GPoH)\right| &= \left|\sum_{v\in V_{n,H}(k)} c_{n,H} \left(\frac{1}{N_n(k)}-\frac{1}{N_{n,H}(k)}\right)\right| \\
&= \frac{|N_{n,H}(k)-N_n(k)|}{N_n(k)N_{n,H}(k)}\sum_{v \in V_{n,H}(k)}c_{n,H}(k)
\end{align*}
Thus, we end up with the upper bound, using that $c_{n,H}(k) \leq 1$ and $N_{n,H}(k) = |V_{n,H}(k)|$,
\begin{align*}
|c(k;\GPo)-c(k;\GPoH)| \leq \frac{X_1}{N_n(k)}+\frac{X_2}{N_n(k)}+\frac{X_3}{N_n(k)}\frac{\sum_{v \in V_{n,H}(k)}c_{n,\leq H}(k)}{N_{n,H}(k)} \leq \frac{X_1+X_2+X_3}{N_n(k)}
\end{align*}
Now using that a.a.s. $N_n(k) \geq a_1(k)n$ gives that a.a.s.
\begin{align*}
|c(k;\GPo)-c(k;\GPoH)| \leq \frac{X_1+X_2+X_3}{a_1(k)n} 
\end{align*}

By Lemma~\ref{lem:n1}, we have that $\E X_1 = \E X_3 = o(n)$ and by Lemma~\ref{lem:n2}, we have that $\E X_2 = o(n)$. We conclude that $\E |c(k;\GPo)-c(k;\GPoH)| = o(1)$. From this, convergence in probability follows by Markov's inequality: for $\epsilon >0$, $\Pee(|c(k;\GPo)-c(k;\GPoH)| \geq \epsilon) \leq \frac{\E|c(k;\GPo)-c(k;\GPoH)|}{\epsilon} = o(1)$.
\end{proof}
\subsubsection{Coupling}
\begin{lemma}\label{lem:coup}
The clustering function in the truncated Poissonized KPKVB model $\GPoH$ and in the truncated box model $\GboxH$ agree, i.e. $c(k;\GPoH) = c(k;\GboxH)$.
\end{lemma}
\begin{proof}
As all vertices involved have height $\leq H$ resp. radial coordinate $\geq R-H$ and as $H\leq \frac{R}{4}$, all vertices involved (in the disk)  will have radial coordinate $\geq \frac{3}{4}R$, such that by Lemma~30(ii) from the paper on the largest component of KPKVB, see Lemma~\ref{lem:coupling_edges}, edges exist between such vertices in the disk if and only if they exist in the box. So the two graphs are isomorphic and hence the clustering functions agree.
\end{proof}

\subsubsection{Deriving the limit integral expression}
\begin{lemma}\label{lem:lim}
Let $\alpha > \frac{1}{2}$, $\nu>0$ and $k\geq 2$ be fixed. Then
$$c(k;\GboxH)\xrightarrow[n\rightarrow\infty]{\Pee}\gamma(k).$$
\end{lemma}
%\begin{proof}
%We will show that $\E c(k;G_{n,Po,H}) \rightarrow \gamma(k)$ as $n\rightarrow\infty$ and $Var(c(k;G_{n,Po,H})) \rightarrow  0$. From there, the claim will follow by Chebychev's inequality: Let $\epsilon>0$, then $$\Pee(|c(k;G_{n,Po,H})-\E c(k;G_{n,Po,H})|\geq \epsilon)\leq \frac{Var(c(k;G_{n,Po,H}))}{\epsilon^2} = o(1),$$
%i.e. $c(k;G_{n,Po,H})-\E c(k;G_{n,Po,H}) \xrightarrow[n\rightarrow\infty]{\Pee} 0$, hence also $c(k;G_{n,Po,H})-\gamma(k) = c(k;G_{n,Po,H})-\E c(k;G_{n,Po,H})+\E c(k;G_{n,Po,H})-\gamma(k) \xrightarrow[n\rightarrow\infty]{\Pee} 0$.
%\end{proof}

Let $\Rcal_H = \Rcal([0,H])$ denote the truncated box.
For a vertex $v$ of the vertex set $V=V(\GboxH)$, writing $\Gamma(v)$ for the set of all neighbours of vertex $v$, let $$h(V,v) = {k \choose 2}^{-1} \sum_{uw \in {\Gamma(v) \choose 2}} \1_{\{uw \in E(\GboxH)\}} \1_{\{ D_{\GboxH}(v)=k\}}.$$
Let $X = \frac{1}{n}\sum_{v \in V(G_{box,H})} h(V(G_{box,H}),v)$. Let $Z = \frac{N_{box,H}(k)}{n} = \frac{1}{n}\sum_{v \in \GboxH} \1_{\{D_{\GboxH}(v)=k\}}$. Then we have that $c(k;\GboxH) = \frac{X}{Z}$. We will show separately that numerator $X$ and denominator $Z$ converge in probability (where the limit of $Z$ is a positive constant). From this, the claim of the lemma will follow as convergence in probability is closed under taking quotients.



By the Campbell-Mecke formula, we have
\begin{align*}
\E X = \frac{1}{n}\int_{\Rcal_H} \E h(V,v) \mu(dv) = \int_0^H \E h(V,(0,y))\alpha e^{-\alpha y} dy
\end{align*}
Then, we condition on that the additional vertex $(0,y) \in \Rcal_H$ has degree $k$,
\begin{align*}
\E X = \int_0^H \E[h(V,(0,y))|D_{G_{box,H}}(0,y)=k]\Pee(D_{G_{box,H}}(0,y)=k)\alpha e^{-\alpha y}dy
\end{align*}
We observe that for $H \rightarrow \infty$, the integrand converges point-wise for $y \in (0,\infty)$ to $P(y)\rho(y,k)\alpha e^{-\alpha y}$ and is always bounded by the integrable function $\alpha e^{-\alpha y}$.
Therefore, by the theorem of dominated convergence, as $H \rightarrow \infty$,
\begin{align*}
\E X \rightarrow \int_0^\infty P(y)\rho(y,k)\alpha e^{-\alpha y}dy
\end{align*}
Note that for $X^2$, the summation can be split into off-diagonal (with $v_1 \not = v_2$) and diagonal (with $v_1=v_2$) terms, i.e. write $X^2 = \frac{1}{n^2}(Y_1+Y_2)$ where 
$$Y_1 = \sum_{v_1 \not = v_2 \in V(G_{box,H})} h(V(G_{box,H}),v_1)h(V(G_{box,H}),v_2)$$ and $Y_2 = \sum_{v_1 \in V(G_{box,H})} h(V(G_{box,H}),v_1)^2$. As $h \leq 1$, $Y_2 \leq n X\leq n$ and so, $\E Y_2 \leq n$, we have by the Campbell-Mecke formula,
\begin{align*}
\E Y_1 = \int_{\Rcal_H} \int_{\Rcal_H} \E[h(V(G_{box,H}),v_1)h(V(G_{box,H}),v_2)]\mu(dv_1)\mu(dv_2)
\end{align*}
If $v_1=(x_1,y_1)$ and $v_2=(x_2,y_2)$ have horizontal distance $>2e^H$, their neighbourhood balls are disjoint w.r.t. the connection rule of the truncated box model $\GboxH$. To see this, assume that $v_3=(x_3,y_3)$ is adjacent to both $v_1$ and $v_2$, then $|x_1-x_3|\leq e^{\frac{1}{2}(y_1+y_3)}$ and $|x_2-x_3|\leq e^{\frac{1}{2}(y_2+y_3)}$. As $y_1, y_2, y_3 \leq H$, it follows that $|x_1-x_3| \leq e^H$ and $|x_2-x_3|\leq e^H$, which implies that $|x_1-x_2| \leq |x_1-x_3|+|x_2-x_3|\leq 2e^H$, contradiction.

Due to the disjointedness of the neighbourhood balls and the properties of the Poisson process $h(V,v_1)$ and $h(V,v_2)$ are independent. The horizontal distance is uniform in $[0,\frac{\pi n}{2\nu}]$. Therefore, the probability that two vertices have horizontal distance $\leq 2e^H$ is $\leq \frac{4\nu e^H}{\pi n} = o(1)$. Writing $d_H$ for the horizontal distance, we can continue the calculation of the second moment,
\begin{align*}
\E Y_1 &= \int_{} \int_{\substack{\Rcal_H\times \Rcal_H:\\ |x_1-x_2|>2 e^H}} \E[ h(V,v_1)]\E[h(V,v_2)]\mu(dv_1)\mu(dv_2)\\
&\qquad+\int_{} \int_{\substack{\Rcal_H\times \Rcal_H:\\ |x_1-x_2|\leq2 e^H}} \E[ h(V,v_1)]\E[h(V,v_2)]\mu(dv_1)\mu(dv_2) \\
&= (1+o(1))\int_{} \int_{\Rcal_H\times \Rcal_H} \E[ h(V,v_1)]\E[h(V,v_2)]\mu(dv_1)\mu(dv_2) \\
&= (1+o(1)) (n\E X)^2%+o((n\E X)^2)+o(n)
\end{align*}
It follows that $\E X^2 = (1+o(1))(\E X)^2$, which implies $Var(X) \rightarrow 0$. We conclude
$$X \xrightarrow[n\rightarrow\infty]{\Pee} \int_0^\infty P(y)\rho(y,k)\alpha e^{-\alpha y}dy.$$
In a similar way, by the Campbell-Mecke formula,
\begin{align*}
\E Z =\frac{1}{n} \int_{\Rcal_H} \Pee(D_{G_{box,H}}(v)=k) \mu(dv) = \int_0^H \Pee(D_{G_{box,H}}(0,y)=k) \alpha e^{-\alpha y}dy
\end{align*}
By the theorem of dominated convergence, it follows that
\begin{align*}
\E Z \rightarrow \int_0^\infty \rho(y,k)\alpha e^{-\alpha y}dy
\end{align*}
The statement that $Var(Z) \rightarrow 0$ follows by the same steps as for $X$, just using $h(V,v) = \1_{\{D_{\GboxH}(v)=k\}}$ instead. Hence,
\begin{align*}
\E Z \xrightarrow[n\rightarrow\infty]{\Pee} \int_0^\infty \rho(y,k)\alpha e^{-\alpha y}dy
\end{align*}
Finally note that 
\begin{align*}
\gamma(k) = \frac{\int_0^\infty P(y)\rho(y,k)\alpha e^{-\alpha y}dy}{\int_0^\infty \rho(y,k)\alpha e^{-\alpha y}dy}.
\end{align*}

\subsubsection{Putting everything together}
\begin{proofof}{Theorem~\ref{thm:mainkfixed}}
The overall statement follows from the lemmas together with the fact that convergence in probability is closed under addition.
Write 
$$c(k;G_n)-\gamma(k) = c(k;G_n) - c(k;\GPo)+c(k;\GPo)-c(k;\GPoH)+c(k;\GPoH)-\gamma(k).$$ 
The first difference converges in probability to zero due to the Poissonization Lemma~\ref{lem:pok}, the second difference due to the truncation Lemma~\ref{lem:trunc}, then for the third difference we use $c(k;\GPoH)=c(k;\GboxH)$ from Lemma~\ref{lem:coup} and then Lemma~\ref{lem:lim} on the derivation of the limit integral.
\end{proofof}

\subsection{Overall clustering coefficient, proving Theorem~\ref{thm:maincc}}
In this subsection, we want to infer that the clustering coefficient of the KPKVB random graph $c(G_n) \xrightarrow{\Pee} \gamma$ by using that $c(k;G_n) \xrightarrow{\Pee} \gamma(k)$ for fixed $k$.

\begin{proofof}{Theorem~\ref{thm:maincc}}
Let $\epsilon >2\delta >0$.
As the $p_k$ form the probability mass function of an $\N_0$-valued random variable, we have that $\sum_{k=0}^\infty p_k = 1$, which implies that there is $K_\delta >0$ such that $1-\delta < \sum_{k=0}^{K_\delta} p_k \leq 1$.
From Equations~\ref{eq:def_pk}, \ref{eq:gammakint} and \ref{eq:gammaint}, we know that $\gamma = \sum_{k=2}^\infty \gamma(k)p_k$. 
From this, it follows that $(1-\delta)\gamma < \sum_{k=2}^{K_\delta} \gamma(k) p_k \leq 1$ and hence that $|\gamma - \sum_{k=2}^{K_\delta} \gamma(k)p_k| \leq \delta$. Write $p(k;G_n)=\frac{N_n(k)}{n}$ for the fraction of degree $k$ vertices in the KPKVB model. From the definition of the clustering coefficient and function, it follows immediately that $c(G_n) = \sum_{k=2}^\infty c(k;G_n)p(k;G_n)$. By the initial observation, it follows again that $(1-\delta)c(k;G_n) < \sum_{k=2}^{K_\delta} c(k;G_n)p(k;G_n) \leq 1$, and hence, $|c(G_n)-\sum_{k=2}^{K_\delta} c(k;G_n)p(k;G_n)| \leq \delta$.
Putting these observations together:
\begin{align*}
|c(G_n)-\gamma| &\leq \left|c(G_n)-\sum_{k=2}^{K_\delta}c(k;G_n)p(k;G_n)\right|+\left|\sum_{k=2}^{K_\delta} c(k;G_n)p(k;G_n) -\sum_{k=2}^{K_\delta} \gamma(k)p_k\right| +\left|\sum_{k=2}^{K_\delta} \gamma(k)p_k-\gamma\right| \\
 &\leq \delta +\left|\sum_{k=2}^{K_\delta} c(k;G_n)p(k;G_n) -\sum_{k=2}^{K_\delta} \gamma(k)p_k\right| +\delta 
\end{align*}
Therefore, $\Pee(|c(G_n)-\gamma|\geq \epsilon) \leq \Pee(\left|\sum_{k=2}^{K_\delta} c(k;G_n)p(k;G_n) -\sum_{k=2}^{K_\delta} \gamma(k)p_k\right| \geq \epsilon - 2\delta) \rightarrow 0$ because $c(k;G_n) \xrightarrow{\Pee} \gamma(k)$ for $k=2,\dots,K_\delta$ and $p(k;G_n) \xrightarrow{\Pee} p_k$ together imply that $\sum_{k=2}^{K_\delta} c(k;G_n)p(k;G_n) \xrightarrow{\Pee} \sum_{k=2}^{K_\delta} \gamma(k) p_k$. %p_k =  = \Theta(k^{-(2\alpha +1)})
\end{proofof}



