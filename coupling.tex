\section{Equivalence for local clustering in $\GPo$ and $\Gbox$}\label{sec:coupling_H_P_n}

In this section we establish the equivalence between $c^\ast(k; G_n)$ and $c^\ast(k; \Gbox)$ as expressed in Proposition~\ref{prop:couling_c_H_P}, using the coupling procedure explained in Section~\ref{ssec:coupling_H_P}. 

Recall that $\mathcal{P}$ denotes a Poisson process on $\mathbb{R} \times \mathbb{R}_+$, with intensity $f(x,y)$, $I_n = \frac{\pi}{2}e^{R_n/2}$, $\mathcal{R} = (-I_n, I_n] \times (0,R_n]$ and $\mathcal{V}_n = \mathcal{P}\cap \mathcal{R}$. In addition we define for any interval $I \subseteq \mathbb{R}_+$, $\Rcal(I) := (-I_n,I_n] \times I$ and denote by $\BallPon{p}$ the \emph{ball}
\[
	\BallPon{p} = \left\{p^\prime \in \mathcal{V}_n : |x - x^\prime |_{\pi e^{R_n/2}} < e^{\frac{y+y^\prime}{2}}\right\}.
\]
Note that when $p \in \mathcal{V}_n$ then $\BallPon{p}$ denotes its neighborhood in the graph $\Gbox$. 
%Note that the above definition implies that for all $y\in [0,R_n]$ we have 
%\begin{equation} \label{eq:upper_ballPo_inclusion}
%\Rcal([R_n-y - 2\ln (\pi/2), R_n]) \subseteq \BallPon{(0,y)}
%\end{equation}
%- this is a fact which we are going to use several times in our analysis. \PvdH{@All: Maybe we should either add figure to illustrate this or we could refer to a previous figure.}

For any Borel-measurable subset $S \subseteq \mathbb{R} \times \mathbb{R}_+$, we let 
\[
	\mu (S) = \int_S f(x,y) \, dx \, dy = \frac{\nu \alpha}{\pi}\int_S e^{-\alpha y}dy.
\]
Thus, the number of points of $\Pcal_{\alpha, \nu}$ inside $S$ is distributed as $\Po ({\mu_{\alpha, \nu,} (S)})$.

Finally, we recall the map $\Psi$ from~\eqref{eq:def_Psi}
\[
	\Psi(r,\theta) = \left(\theta \frac{e^{R_n/2}}{2}, R_n - r\right),
\] 
and remind the reader that $\BallHyp{p}$ denotes the image under $\Psi$ of the ball of hyperbolic radius $R_n$ around the point $\Psi^{-1}(p)$ and that under the coupling between the hyperbolic random graph and the finite box model, described in Section~\ref{ssec:coupling_H_P}, two points $p$ and $p^\prime$ are connected if and only if
\[
	|x-x^\prime|_{\pi e^{r_n/2}} \le \Phi(R_n - y, R_n - y^\prime),
\]
where the function $\Phi$ can be approximated, for $y + y^\prime < R_n$, using Lemma~\ref{lem:asymptotics_Omega_hyperbolic} by  
\[
	e^{\frac{1}{2}(y+y^\prime)} - K e^{\frac{3}{2}(y+y^\prime) - R_n} \leq \Phi(R_n - y, R_n - y^\prime) 
		\leq  e^{\frac{1}{2}(y+y^\prime)} + K e^{\frac{3}{2}(y+y^\prime) - R_n}.
\]

%To prove Proposition~\ref{prop:couling_c_H_P} we calculate the error in two steps. First we show in Section~\ref{ssec:coupling_HP_ast_P} that
%\[
%	\lim_{n \to \infty} s(k_n) \Exp{\left|c^\ast(k_n; \GPo) - c_{\Pcal,n}^\ast(k_n)\right|} = 0,
%\]
%Then, in Section~\ref{ssec:coupling_H_HP}, we prove Proposition~\ref{prop:clustering_ast_H_Pois}, i.e.
%\[
%	\lim_{n \to \infty} s(k_n) \Exp{\left| c^\ast(k_n; G_n) - c^\ast(k_n; \GPo)\right|} = 0.
%\]
%Together these results yield Proposition~\ref{prop:couling_c_H_P}.

\subsection{Some results on the hyperbolic geometric graph}

We start with some basic results for the hyperbolic random geometric graph. Observe that~\eqref{eq:asymp2} from Lemma~\ref{lem:asymptotics_Omega_hyperbolic} implies the following. 
\begin{corollary}\label{cor:balls_inclusion}
\begin{equation*}
 \BallPo{p} \cap \Rcal ([K,R_n]) \subseteq \BallHyp{p} \cap \Rcal ([K,R_n]). 
\end{equation*}
\end{corollary}


Furthermore, Lemma~\ref{lem:asymptotics_Omega_hyperbolic} enables us to determine the measure of a ball around a given point $p=(0,y)$ - this is will be fairly useful in our subsequent analysis. 

Let $p \in \Rcal$. Then we can see that the curve $x^\prime = e^{\frac{1}{2} (y + y^\prime)}$ with $x^\prime \geq 0$ meets the right boundary of $\Rcal_N$, that is, the line $x^\prime = \frac{\pi}{2} e^{R_n/2}$ at $y^\prime = R_n - y + 2\ln \frac{\pi}{2}$. Hence, any point $p^{\prime} \in \Rcal ([R_n - y + 2\ln \frac{\pi}{2}, R_n])$ is included in $\BallPo{p}$. In other words,
\begin{equation*} \label{eq:P_ball_inclusion_lower}
\BallPo{p} \cap \Rcal ([R_n - y +2\ln \frac{\pi}{2},R_n]) = \Rcal ([R_n - y + 2\ln \frac{\pi}{2},R_n]).
\end{equation*}
This together with the fact that for any $u^\prime = (r^\prime, \theta^\prime)$,
\[
	r^\prime < y = R_n -r \Rightarrow d_\H(\Psi^{-1}(p),u^\prime) \le R_n
\]
implies that 
\begin{equation}\label{eq:symm_diff_upper_P} 
(\BallHyp{p} \bigtriangleup \BallPo{p})  \cap \Rcal ([R_n - y + 2 \ln \frac{\pi}{2},R_n]) = \emptyset,
\end{equation}
where $A \bigtriangleup B$ denotes the symmetric difference. We can now compute the expected number of points in $\BallHyp{p} \bigtriangleup \BallPo{p}$, i.e. those that belong are a neighbor of $p$ in only one of the two models.

\begin{lemma}\label{lem:sym_diff_measure_H_P}
Let $0 \le y_n <R_n$ be such that $R_n - y_n \to \infty$ and write $p_n = (x_n, y_n)$. Then we have, as $n \to \infty$,
\[
	\mu (\BallHyp{p_n} \bigtriangleup \BallPo{p_n} ) 
	= \Theta (1) \cdot \begin{cases} 
		e^{(1/2-\alpha)R_n + \alpha y_n}, & \mbox{if } \alpha < 3/2 \\
		(R_n-y_n) e^{3y/2 - R_n}, & \mbox{if }\alpha = 3/2\\
		e^{3y_n/2 - R_n}, &  \mbox{if } \alpha > 3/2 
	\end{cases}.
\]
\end{lemma}

\begin{proof}
Let $r_n := R_n - y$. Lemma~\ref{lem:asymptotics_Omega_hyperbolic} implies that for such a $p_n$, if a point $p$ belongs to $\BallHyp{p_n} \bigtriangleup \BallPo{p_n} \cap \Rcal ([0,r_n])$ then 
\[
	|x_n - x| = \Theta(1) \cdot e^{\frac{3}{2} (y_n + y) - R_n}.
\]
Now, if $p \in [r_n, r_n + 2 \ln \frac{\pi}{2})]$ and also $p \in \BallHyp{p_n} \bigtriangleup \BallPo{p_n}$, then 
\[
	|x_n-x| =\frac{\pi}{2} e^{R_n/2} - e^{\frac{1}{2} (y_n + y)}.
\]
Finally,~\eqref{eq:symm_diff_upper_P} implies that no point in $\Rcal ([r_n+2\ln \frac{\pi}{2},R_n])$ belongs to $\BallHyp{p_n} \bigtriangleup \BallPo{p_n}$. We first compute the expected number of points in $p \in \BallHyp{p_n} \bigtriangleup \BallPo{p_n}$ that have $R_n - y \le r_n$. The result depends on the value of $\alpha$, yielding the following three cases
\begin{align*}
	\mu (\BallHyp{p_n} \bigtriangleup \BallPo{p_n} \cap \Rcal ([0,r_n])) 
	&= \Theta(1) \cdot e^{3y_n/2 - R_n}\int_0^{r_n} e^{(3/2-\alpha) y} \, dy \\
	&= \Theta(1)\cdot \begin{cases} e^{(1/2-\alpha)R_n + \alpha y_n}, & \mbox{if } \alpha < 3/2 \\
		(R_n - y_n) e^{3y_n/2 - R_n}, & \mbox{if }\alpha = 3/2\\
		e^{3y_n/2 - R_n}, &  \mbox{if } \alpha > 3/2
	\end{cases}.
\end{align*}
Next we compute the number of remaining points in $\BallHyp{p_n} \bigtriangleup \BallPo{p_n}$, 
\begin{align*}
	\mu (\BallHyp{p_n} \bigtriangleup \BallPo{p_n} \cap \Rcal ([r_n, R_n])) 
	&= \frac{\nu\alpha}{\pi} \int_{r_n}^{r_n + 2 \ln \frac{\pi}{2}} 
		\left(\frac{\pi}{2} e^{R_n/2} - e^{\frac{1}{2} (y_n + y)}\right)e^{-\alpha y} \, dy \\
	&= O(1) \cdot e^{R_n/2} \int_{r_n}^{r_n + 2 \ln \frac{\pi}{2}} e^{-\alpha y} \, dy 
		= O(1) \cdot e^{R_n/2} e^{-\alpha r_n} \\
	&=O(1) \cdot e^{(1/2 - \alpha) R_n + \alpha y_n}.
\end{align*}
Now note that for any $\alpha > 3/2$, we have 
\[
	\left( (1/2 - \alpha) R_n + \alpha y_n\right) - \left(3y_n/2 - R_n \right)
	= (3/2 -\alpha) (R_n- y_n) \to -\infty,
\]
by our assumption on $y_n$. For $\alpha = 3/2$, these two quantities are equal. From these observations, we deduce that 
\[
	\mu (\BallHyp{p_n} \bigtriangleup \BallPo{p_n} ) 
	= \Theta (1) \cdot \begin{cases} 
		e^{(1/2-\alpha)R_n + \alpha y_n}, & \mbox{if } \alpha < 3/2 \\
		r_n e^{3y_n/2 - R_n}, & \mbox{if }\alpha = 3/2\\
		e^{3y_n/2 - R_n}, &  \mbox{if } \alpha > 3/2 
	\end{cases}.
\]
\end{proof}

%\subsection{Figures for proof strategy}
%
%\PvdH{I will create some figures illustrating the proof strategy for the coupling. These will be merged with the computations as soon as Nikolaos finishes the computations.}
%
%\begin{figure}
%
%%\begin{tikzpicture}
%%
%%	\pgfmathsetmacro{\u}{0}
%%	\pgfmathsetmacro{\v}{0.5}
%%	\pgfmathsetmacro{\uu}{1}
%%	\pgfmathsetmacro{\vv}{1.2}
%%	\pgfmathsetmacro{\uuu}{-3}
%%	\pgfmathsetmacro{\vvv}{2}
%%	\pgfmathsetmacro{\e}{0.05}
%%	\pgfmathsetmacro{\K}{1}
%%	\pgfmathsetmacro{\R}{3}
%%	\pgfmathsetmacro{\r}{(pi/2)*exp(\R/2)}
%%
%%	\draw[line width=1pt,dashed] (-\r,0) -- (\r,0) -- (\r,\R) -- (-\r,\R) -- (-\r,0);
%%
%%	\draw[domain=0:{\R-\v},variable=\x, samples=200, black, line width=1pt] plot ({(exp(\R/2)/2)*rad(acos((cosh(\R-\v)*cosh(\R-\x)-cosh(\R))/(sinh(\R-\v)*sinh(\R-\x))))},{\x});
%%
%%\end{tikzpicture}
%
%\begin{tikzpicture}[scale=0.85]
%	%Define the coordinates 
%	%p = (\u,\v), p_1 = (\uu, \vv)
%	%Box \Rcal has width 2\r and height \t (\r = \pi/2 e^{R_n/2})
%	\pgfmathsetmacro{\u}{0}
%	\pgfmathsetmacro{\v}{0.5}
%	\pgfmathsetmacro{\uu}{1}
%	\pgfmathsetmacro{\vv}{1.2}
%	\pgfmathsetmacro{\uuu}{5.5}
%	\pgfmathsetmacro{\vvv}{1.7}
%	\pgfmathsetmacro{\epsilon}{0.05}
%	\pgfmathsetmacro{\K}{1}
%	\pgfmathsetmacro{\R}{3}
%	\pgfmathsetmacro{\r}{(pi/2)*exp(\R/2)}
%	\pgfmathsetmacro{\t}{\R}
%	
%	\pgfmathsetmacro{\leftintvandvvv}{-exp((\v + \vvv)/2)}
%	\pgfmathsetmacro{\rightintvandvvv}{exp((\v + \vvv)/2)}
%	\pgfmathsetmacro{\leftintvvandvvv}{\uu-exp((\vv + \vvv)/2)}
%	\pgfmathsetmacro{\rightintvvandvvv}{\uu+exp((\vv + \vvv)/2)}
%		
%	%The box \Rcal
%	\draw[line width=1pt,dashed] (0,0) -- (\r,0) -- (\r,\t) -- (0,\t);
%
%	%Dram all three nodes
%    \draw node[fill, circle, inner sep=0pt, minimum size=5pt] (p0) at (\u,\v) {};
%    \path (p0)+(-0.3,0.3) node {$p_0$};
%    
%    \draw node[fill,blue, circle, inner sep=0pt, minimum size=5pt] (p1) at (\uu,\vv) {};
%    \path (p1)+(-0.3,0.3) node {\color{blue}$p_1$};	
%
%	
%	%Boundaries p_0 = (\u,\v)
%	
%	%Limit model
%	%Right boundary
%	\pgfmathsetmacro{\rightbounduv}{\u+exp((\v)/2)}
%	\draw[domain=0:\R,smooth,variable=\y,black, dotted, line width=1pt] plot ({exp((\v+\y)/2)},{\y});
%    
%    %Hyperbolic model
%    %Right boundary
%	\draw[domain=0:{\R-\v},variable=\y, samples=200, black, line width=1pt] plot 		
%		({(exp(\R/2)/2)*rad(acos((cosh(\R-\v)*cosh(\R-\y)-cosh(\R))/(sinh(\R-\v)*sinh(\R-\y))))},{\y});
%
%    
%	%Boundaries p_1 = (\uu,\vv)
%	
%	%Limit model
%	\pgfmathsetmacro{\rboundvv}{2*ln(\r-\uu)-\vv}
%	\pgfmathsetmacro{\lboundvv}{2*ln(\r+\uu)-\vv}
%
%    
%    %Hyperbolic model
%	\pgfmathsetmacro{\rhboundvv}{1.76}
%	\pgfmathsetmacro{\lhboundvv}{\R-\vv}
%
%	%Intersection
%	\pgfmathsetmacro{\regionright}{\uu+exp((\vv+\rboundvv)/2)}
%
%	\draw[red,line width=1pt,pattern=north west lines, pattern color=red] 
%		plot[domain=0.6:\rboundvv,smooth,variable=\y,red] ({\uu+exp((\vv+\y)/2)},{\y}) 
%		--
%		(\regionright,\rboundvv) 
%		--
%		(\regionright,\rhboundvv)
%		--
%		plot[domain={\rhboundvv-0.01}:0.6,smooth,variable=\y,red] ({\uu+(exp(\R/2)/2)*rad(acos((cosh(\R-\vv)*cosh(\R-\y)-cosh(\R))/(sinh(\R-\vv)*sinh(\R-\y))))},{\y});
%		
%	\draw[red,line width=1pt,pattern=north west lines, pattern color=red] 
%		plot[domain=0:0.6,smooth,variable=\y,red] ({\uu+exp((\vv+\y)/2)},{\y}) 
%		--
%		plot[domain=0.6:0,smooth,variable=\y,red] ({\uu+(exp(\R/2)/2)*rad(acos((cosh(\R-\vv)*cosh(\R-\y)-cosh(\R))/(sinh(\R-\vv)*sinh(\R-\y))))},{\y});
%
%	%Limit model
%	%Right boundary
%	\draw[domain=0:\rboundvv,smooth,variable=\y,blue, dotted, line width=1pt] plot ({\uu+exp((\vv+\y)/2)},{\y});
%
%    %Hyperbolic model
%    %Right boundary
%	\draw[domain=0:\rhboundvv,variable=\y, samples=200, blue, line width=1pt] plot 		
%		({\uu+(exp(\R/2)/2)*rad(acos((cosh(\R-\vv)*cosh(\R-\y)-cosh(\R))/(sinh(\R-\vv)*sinh(\R-\y))))},{\y});
%
%
%
%\end{tikzpicture}~\hspace{5pt}~
%\begin{tikzpicture}[scale=0.85]
%	%Define the coordinates 
%	%p = (\u,\v), p_1 = (\uu, \vv)
%	%Box \Rcal has width 2\r and height \t (\r = \pi/2 e^{R_n/2})
%	\pgfmathsetmacro{\u}{0}
%	\pgfmathsetmacro{\v}{0.5}
%	\pgfmathsetmacro{\uu}{1}
%	\pgfmathsetmacro{\vv}{2.7}
%	\pgfmathsetmacro{\uuu}{5.5}
%	\pgfmathsetmacro{\vvv}{1.7}
%	\pgfmathsetmacro{\epsilon}{0.05}
%	\pgfmathsetmacro{\K}{1}
%	\pgfmathsetmacro{\R}{3}
%	\pgfmathsetmacro{\r}{(pi/2)*exp(\R/2)}
%	\pgfmathsetmacro{\t}{\R}
%	
%	\pgfmathsetmacro{\leftintvandvvv}{-exp((\v + \vvv)/2)}
%	\pgfmathsetmacro{\rightintvandvvv}{exp((\v + \vvv)/2)}
%	\pgfmathsetmacro{\leftintvvandvvv}{\uu-exp((\vv + \vvv)/2)}
%	\pgfmathsetmacro{\rightintvvandvvv}{\uu+exp((\vv + \vvv)/2)}
%		
%	%The box \Rcal
%	\draw[line width=1pt,dashed] (0,0) -- (\r,0) -- (\r,\t) -- (0,\t);
%
%	%Dram all three nodes
%    \draw node[fill, circle, inner sep=0pt, minimum size=5pt] (p0) at (\u,\v) {};
%    \path (p0)+(-0.3,0.3) node {$p_0$};
%    
%    \draw node[fill,blue, circle, inner sep=0pt, minimum size=5pt] (p1) at (\uu,\vv) {};
%    \path (p1)+(-0.3,-0.3) node {\color{blue}$p_1$};	
%
%	
%	%Boundaries p_0 = (\u,\v)
%	
%	%Limit model
%	%Right boundary
%	\pgfmathsetmacro{\rightbounduv}{\u+exp((\v)/2)}
%	\draw[domain=0:\R,smooth,variable=\y,black, dotted, line width=1pt] plot ({exp((\v+\y)/2)},{\y});
%
%    
%    %Hyperbolic model
%    %Right boundary
%	\draw[domain=0:{\R-\v},variable=\y, samples=200, black, line width=1pt] plot 		
%		({(exp(\R/2)/2)*rad(acos((cosh(\R-\v)*cosh(\R-\y)-cosh(\R))/(sinh(\R-\v)*sinh(\R-\y))))},{\y});
%
%    
%	%Boundaries p_1 = (\uu,\vv)
%	
%	%Limit model
%	\pgfmathsetmacro{\rboundvv}{2*ln(\r-\uu)-\vv}
%	\pgfmathsetmacro{\lboundvv}{2*ln(\r+\uu)-\vv}
%
%    
%    %Hyperbolic model
%	\pgfmathsetmacro{\rhboundvv}{0.277}
%	\pgfmathsetmacro{\lhboundvv}{\R-\vv}
%
%	%Intersection
%	\pgfmathsetmacro{\regionright}{\uu+exp((\vv+\rboundvv)/2)}
%	\pgfmathsetmacro{\intersection}{0.15}
%	
%	\draw[red,line width=1pt,pattern=north west lines, pattern color=red] 
%		plot[domain=\intersection:\rboundvv,smooth,variable=\y,red] ({\uu+exp((\vv+\y)/2)},{\y}) 
%		--
%		(\regionright,\rboundvv) 
%		--
%		(\regionright,\rhboundvv)
%		--
%		plot[domain={\rhboundvv-0.01}:\intersection,smooth,variable=\y,red] ({\uu+(exp(\R/2)/2)*rad(acos((cosh(\R-\vv)*cosh(\R-\y)-cosh(\R))/(sinh(\R-\vv)*sinh(\R-\y))))},{\y});
%		
%	\draw[red,line width=1pt,pattern=north west lines, pattern color=red] 
%		plot[domain=0:\intersection,smooth,variable=\y,red] ({\uu+exp((\vv+\y)/2)},{\y}) 
%		--
%		plot[domain=\intersection:0,smooth,variable=\y,red] ({\uu+(exp(\R/2)/2)*rad(acos((cosh(\R-\vv)*cosh(\R-\y)-cosh(\R))/(sinh(\R-\vv)*sinh(\R-\y))))},{\y});
%
%	%Limit model
%	%Right boundary
%	\draw[domain=0:\rboundvv,smooth,variable=\y,blue, dotted, line width=1pt] plot ({\uu+exp((\vv+\y)/2)},{\y});
%
%    %Hyperbolic model
%    %Right boundary
%	\draw[domain=0:\rhboundvv,variable=\y, samples=200, blue, line width=1pt] plot 		
%		({\uu+(exp(\R/2)/2)*rad(acos((cosh(\R-\vv)*cosh(\R-\y)-cosh(\R))/(sinh(\R-\vv)*sinh(\R-\y))))},{\y});
%
%
%\end{tikzpicture}
%
%\begin{tikzpicture}[scale=0.85]
%	%Define the coordinates 
%	%p = (\u,\v), p_1 = (\uu, \vv) and p_2 = (\uuu, \vvv)
%	%Box \Rcal has width 2\r and height \t (\r = \pi/2 e^{R_n/2})
%	\pgfmathsetmacro{\u}{0}
%	\pgfmathsetmacro{\v}{0.5}
%	\pgfmathsetmacro{\uu}{1}
%	\pgfmathsetmacro{\vv}{1.2}
%	\pgfmathsetmacro{\uuu}{3}
%	\pgfmathsetmacro{\vvv}{2.5}
%	\pgfmathsetmacro{\epsilon}{0.05}
%	\pgfmathsetmacro{\K}{1}
%	\pgfmathsetmacro{\R}{3}
%	\pgfmathsetmacro{\r}{(pi/2)*exp(\R/2)}
%	\pgfmathsetmacro{\t}{\R}
%	
%	\pgfmathsetmacro{\leftintvandvvv}{-exp((\v + \vvv)/2)}
%	\pgfmathsetmacro{\rightintvandvvv}{exp((\v + \vvv)/2)}
%	\pgfmathsetmacro{\leftintvvandvvv}{\uu-exp((\vv + \vvv)/2)}
%	\pgfmathsetmacro{\rightintvvandvvv}{\uu+exp((\vv + \vvv)/2)}
%		
%	%The box \Rcal
%	\draw[line width=1pt,dashed] (0,0) -- (\r,0) -- (\r,\t) -- (0,\t);
%
%	%Dram all three nodes
%    \draw node[fill, circle, inner sep=0pt, minimum size=5pt] (p0) at (\u,\v) {};
%    \path (p0)+(-0.3,0.3) node {$p_0$};
%    
%    \draw node[fill,blue, circle, inner sep=0pt, minimum size=5pt] (p1) at (\uu,\vv) {};
%    \path (p1)+(-0.3,-0.3) node {\color{blue}$p_1$};	
%
%	
%	%Boundaries p_0 = (\u,\v)
%	
%	%Limit model
%	%Right boundary
%	\pgfmathsetmacro{\rightbounduv}{\u+exp((\v)/2)}
%	\draw[domain=0:\R,smooth,variable=\y,black, dotted, line width=1pt] plot ({exp((\v+\y)/2)},{\y});
%
%    
%    %Hyperbolic model
%    %Right boundary
%	\draw[domain=0:{\R-\v},variable=\y, samples=200, black, line width=1pt] plot 		
%		({(exp(\R/2)/2)*rad(acos((cosh(\R-\v)*cosh(\R-\y)-cosh(\R))/(sinh(\R-\v)*sinh(\R-\y))))},{\y});
%
%    
%	%Boundaries p_1 = (\uu,\vv)
%	
%	%Limit model
%	\pgfmathsetmacro{\rboundvv}{2*ln(\r-\uu)-\vv}
%	\pgfmathsetmacro{\lboundvv}{2*ln(\r+\uu)-\vv}
%
%    
%    %Hyperbolic model
%	\pgfmathsetmacro{\rhboundvv}{1.76}
%	\pgfmathsetmacro{\lhboundvv}{\R-\vv}
%
%	%Intersection
%	\pgfmathsetmacro{\regionright}{\uu+exp((\vv+\rboundvv)/2)}
%	\pgfmathsetmacro{\intersection}{0.6}
%	
%	\draw[red,line width=1pt,pattern=north west lines, pattern color=red] 
%		plot[domain=\intersection:\rboundvv,smooth,variable=\y,red] ({\uu+exp((\vv+\y)/2)},{\y}) 
%		--
%		(\regionright,\rboundvv) 
%		--
%		(\regionright,\rhboundvv)
%		--
%		plot[domain={\rhboundvv-0.01}:\intersection,smooth,variable=\y,red] ({\uu+(exp(\R/2)/2)*rad(acos((cosh(\R-\vv)*cosh(\R-\y)-cosh(\R))/(sinh(\R-\vv)*sinh(\R-\y))))},{\y});
%		
%	\draw[red,line width=1pt,pattern=north west lines, pattern color=red] 
%		plot[domain=0:\intersection,smooth,variable=\y,red] ({\uu+exp((\vv+\y)/2)},{\y}) 
%		--
%		plot[domain=\intersection:0,smooth,variable=\y,red] ({\uu+(exp(\R/2)/2)*rad(acos((cosh(\R-\vv)*cosh(\R-\y)-cosh(\R))/(sinh(\R-\vv)*sinh(\R-\y))))},{\y});
%
%	%Limit model
%	%Right boundary
%	\draw[domain=0:\rboundvv,smooth,variable=\y,blue, dotted, line width=1pt] plot ({\uu+exp((\vv+\y)/2)},{\y});
%
%    %Hyperbolic model
%    %Right boundary
%	\draw[domain=0:\rhboundvv,variable=\y, samples=200, blue, line width=1pt] plot 		
%		({\uu+(exp(\R/2)/2)*rad(acos((cosh(\R-\vv)*cosh(\R-\y)-cosh(\R))/(sinh(\R-\vv)*sinh(\R-\y))))},{\y});
%
%
%\end{tikzpicture}
%\end{figure}



\subsection{Equivalence clustering $\GPo$ and $\Gbox$}\label{ssec:coupling_HP_ast_P}


Here we prove Proposition~\ref{prop:couling_c_H_P}. We first establish a few results regarding the number of nodes of degree $k_n$ in both the Poissonized KPKVB graph $\GPo$ and the finite box model $\Gbox$.

\begin{lemma}\label{lem:N_k_HP_P}
Let $\alpha > 1/2$, $\nu > 0$ and $(k_n)_{n\ge 1}$ be a sequence such that $k_n = \bigO{n^{1/(2\alpha + 1)}}$. Then
\begin{equation} \label{eq:n_k_Hyp}
	\Exp{N_{\Po}(k_n)} = \bigT{1} n k_n^{-(2\alpha + 1)},
\end{equation}
and
\begin{equation} \label{eq:n_k_Po}
	\Exp{N_{\text{box}}(k_n)} = \bigT{1} n k_n^{-(2\alpha + 1)}.
\end{equation}
Moreover,
\begin{equation}\label{eq:equivalence_N_HP_P}
	\lim_{n \to \infty} \frac{\Exp{N_{\Po}(k_n)}}{\Exp{N_{\text{box}}(k_n)}} = 1.
\end{equation}
\end{lemma}

\begin{proof}
By the Campbell-Mecke formula
\[
	\Exp{N_{\Po}(k_n)} = \int_{\Rcal} \rho_{\Po}(y,k_n) f(x,y) \dd x \dd y. 
\]
Then by Lemma~\ref{lem:concentration_argument_rho_approximation} 
\begin{align*}
	\Exp{N_{\Po}(k_n)} &= (1 + \smallO{1})\int_{\Rcal} \rho(y,k_n) f(x,y) \dd x \dd y\\
	&= (1 + \smallO{1}) n \int_{0}^{R_n} \rho(y,k_n) f(x,y) \dd x \dd y = \bigT{1}n k_n^{-(2\alpha + 1)}.
\end{align*}
Similarly,
\[
	\Exp{N_{\text{box}}(k_n)} = (1 + \smallO{1})\int_{\Rcal} \rho(y,k_n) f(x,y) \dd x \dd y
\]
From which the results follow.
\end{proof}

Recall that Proposition~\ref{prop:couling_c_H_P} states
\[
	\lim_{n \to \infty} s(k_n)^{-1} \, \Exp{\left|c^\ast(k_n; \GPo) - c^\ast(k_n;\Gbox)\right|} = 0.
\]

Next recall the definition of $\Kcal_{C}(k_n)$
\[
	\Kcal_C(k_n) = \left\{y \in \R_+ : \frac{k_n - C \sqrt{k_n \log(k_n)}}{\xi} \vee 0 \le e^{\frac{y}{2}}
	\le \frac{k_n + C \sqrt{k_n \log(k_n)}}{\xi} \wedge e^{R_n/2} \right\},
\]
and~\eqref{eq:def_tilde_c_box}
\[
	\widetilde{c}_{\text{box}}(k_n) = \frac{\widetilde{T}_{\text{box}}(k_n,C)}{\binom{k_n}{2}\Exp{N_{\text{box}}(k_n)}},
\]
where $\widetilde{T}_{\text{box}}(k_n,C)$ counts for all nodes $p = (x,y)$ with $y \in \Kcal_{C}(k_n)$ the pairs $(p_1,p_2)$ that form a triangle with $p$, with the exception that it considers $p_2 \in \BallPo{p_1} \cap \Rcal$ instead of $\BallPon{p_1}$. Then using Corollary~\ref{cor:c_ast_box_2_tilde_c_box} we get
\[
	\Exp{\left|c^\ast(k_n; \GPo) - c^\ast(k_n;\Gbox)\right|}
	\le \Exp{\left|c^\ast(k_n; \GPo) - \widetilde{c}_{\text{box}}(k_n)\right|} 
	+ \smallO{s(k_n)},
\]
and hence it is enough to prove that
\[
	\lim_{n \to \infty} s(k_n)^{-1}\Exp{\left|c^\ast(k_n; \GPo) - \widetilde{c}_{\text{box}}(k_n)\right|} = 0.
\]

%Since for $\alpha > 3/4$, $s_{3/4}(k_n) = \log(k_n)^{-1} s_\alpha(k_n) = \smallO{s_\alpha(k_n)}$ it suffices to prove the following two cases:
%\begin{enumerate}
%\item if $1/2 < \alpha \leq 3/4$, then
%\[
%	\lim_{n\to \infty} k_n^{4\alpha -2}\cdot \Exp{\left|c^\ast(k_n; \GPo) - \widetilde{c}_{\text{box}}(k_n)\right|}=0,
%\]
%\item if $3/4 < \alpha$, then
%\[ 
%	\lim_{n\to \infty} k_n \cdot \Exp{\left|c^\ast(k_n; \GPo) - \widetilde{c}_{\text{box}}(k_n)\right|}=0.
%\]
%\end{enumerate}


The following lemma will be frequently used in the proof of Proposition~\ref{prop:couling_c_H_P}.

\begin{lemma} \label{lem:gamma_approx}
Let $t, r \in \mathbb{R}$ be fixed and let $\hat{\rho}(y,k)$ be any of the three probability functions $\rho_{\Po}(y,k), \rho_{\text{box}}(y,k)$ or $\rho(y,k)$. Then for any sequence $k_n$ of positive integers with $k_n = \bigO{n^{\frac{1}{2\alpha + 1}}}$ and $C > 0$ large enough,
\[
	\int_{\Kcal_{C}} e^{t y} \hat{\rho}_n(y,k_n-r) e^{-\alpha y} \dd y = \bigO{1} k_n^{-2\alpha - 1 + 2t}
\]
as $n \to \infty$.
\end{lemma}
\begin{proof}
Note that on $\Kcal_{C}(k_n)$ we have that $e^{ty} = \bigT{k_n^{2t}}$. Hence, by the second statement of Lemma~\ref{lem:concentration_argument_rho_approximation}
\begin{align*}
	\int_{\Kcal_{C}} e^{t y} \hat{\rho}_n(y,k_n-r) e^{-\alpha y} \dd y
	&= \bigT{k_n^{2t}} \int_{\Kcal_{C}} \hat{\rho}_n(y,k_n-r) e^{-\alpha y} \dd y\\
	&= \bigO{k_n^{2t}} (k_n-r)^{-(2\alpha + 1)} = \bigO{1} k_n^{-2\alpha - 1 + 2t}.
\end{align*}
\end{proof}

\begin{proof}[Proof of Proposition~\ref{prop:couling_c_H_P}] 

To keep notations concise we abbreviate $\Exp{N_{\Po}(k_n)}$ and $\Exp{N_{\infty}(k_n)}$ by $\expH$ and $\expP$, respectively. We will also suppress the subscripts $n$ in most expression regarding the graphs $\GPo$ and $\Gbox$. Finally we will write 
\[
	T_{\Po}(p) = \sum_{(p_1, p_2) \in \Pcal \setminus p}^{\ne} T_{\Po}(p,p_1,p_2),
\] 
with
\[
	T_{\Po}(p,p_1,p_2) = \ind{p_1 \in \BallHyp{p}} \ind{p_2 \in \BallHyp{p}} \ind{p_2 \in \BallHyp{p_1}}
\]
to denote the triangle count function for $p$ in $\GPo$. Then we have
\begin{align*} 
	&\Exp{\left|  c^\ast(k_n; \GPo) - \widetilde{c}_{\text{box}}(k_n)\right|}= 
	\binom{k_n}{2}^{-1}\Exp{\left|\sum_{p \in \Pcal} 
    	\frac{\ind{\mathrm{deg}_{\Po}(p) = k_n}}{\expH} T_{\Po}(p)
        - \frac{\ind{\mathrm{deg}_{\infty}(p) = k_n}}{\expP}  \widetilde{T}_{\text{box}}(p)\right|} \\
    &\le\binom{k_n}{2}^{-1} \expH^{-1} \Exp{\left|\sum_{p \in \Pcal} \ind{\mathrm{deg}_{\Po}(y) = k_n} T_{\Po}(p) 
    	- \ind{\mathrm{deg}_{\infty}(p) = k_n} \widetilde{T}_{\text{box}}(p)\right|} \\
    &\hspace{10pt}+ \binom{k_n}{2}^{-1} \left|\frac{1}{\expH} - \frac{1}{\expP}\right|\Exp{
        	\sum_{p \in \Pcal} \ind{\mathrm{deg}_{\infty}(p) = k_n} \widetilde{T}_{\text{box}}(p) }
\end{align*}
The last term can be rewritten as
\begin{align*}
	\left|1 - \frac{\expH}{\expP}\right| \Exp{\widetilde{c}_{\text{box}}(k_n)} = \left|1 - \frac{\expH}{\expP}\right|\gamma(k_n)(1+o(1)),
\end{align*}
where we used Proposition~\ref{prop:convergence_average_clustering_P_n} (See Section~\ref{sec:clustering_Pn_to_P}). The first term in this product converges to zero by Lemma~\ref{lem:N_k_HP_P} while the second term scales as $s(k_n)$. Hence
\[
	\left|1 - \frac{\expH}{\expP}\right| \Exp{\widetilde{c}_{\text{box}}(k_n)} = \smallO{s(k_n)},
\]
and therefore we are left to analyze the other term. By the Campbell-Mecke formula we have that
\begin{align*}
	    &\Exp{\left|\sum_{p \in \Pcal} \ind{\mathrm{deg}_{n}(y) = k_n} T_{\Po}(p) 
	        	- \ind{\mathrm{deg}_{\infty}(p) = k_n} \widetilde{T}_{\text{box}}(p)\right|} \\
	    &= \int_{\Rcal} 
	        \Exp{\left|\ind{\mathrm{deg}_{n}(y) = k_n} T_{\Po}(y)
	        - \ind{\mathrm{deg}_{\infty}(y) = k_n} \widetilde{T}_{\text{box}}(p)\right|} 
	        	f(x,y) \dd y \dd x.
\end{align*}
Since 
\begin{align*}
	\Exp{\frac{\ind{\mathrm{deg}_{n}(y) = k_n}}{\expH} T_{\Po}(y)}
	&\le \binom{k_n}{2} \rho_{\Po}(y,k_n)\expH^{-1} \\
	&= \binom{k_n}{2} \rho_{\Po}(y,k_n)\bigT{\expP^{-1}}\\
	&= \bigT{n^{-1} k_n^{2\alpha + 3}}\rho_{\Po}(y,k_n)
\end{align*}
and similar for the other term, it follows that
\begin{align*}
	&\hspace{-60pt}\Exp{ \left| \frac{\ind{\mathrm{deg}_{n}(y) = k_n}}{\expH} T_{\Po}(y)
		- \frac{\ind{\mathrm{deg}_{\infty}(y) = k_n}}{\expH}  \widetilde{T}_{\text{box}}(p)\right|} \\
	&\le \bigT{n^{-1} k_n^{2\alpha + 3}}\left(\rho_{\Po}(k,n) + \rho_{\text{box}}(y,k_n)\right).
\end{align*}
Therefore, by a concentration of heights argument (c.f. first statement of Lemma~\ref{lem:concentration_argument_rho_approximation}), it is enough to consider the integral
\begin{equation} \label{eq:expectation_total}
	n \int_{\Kcal_{C}(k_n)} \Exp{\left|\ind{\mathrm{deg}_{n}(y) = k_n} T_{\Po}(y)
		- \ind{\mathrm{deg}_{\infty}(y) = k_n} P(y)\right|} e^{-\alpha y} \dd y,
\end{equation}
where we also used that $f(x,y)$ is simply a constant multiple of the function $e^{-\alpha y}$. Since $\expH
= \bigT{n k_n^{-(2\alpha + 1)}}$ we have to show that
\[
	\lim_{n \to \infty} k_n^{2\alpha - 1} s(k_n)^{-1} \int_{\Kcal_{C}(k_n)} 
		\Exp{\left|\ind{\mathrm{deg}_{n}(y) = k_n} T_{\Po}(y)
			- \ind{\mathrm{deg}_{\infty}(y) = k_n} P(y)\right|} e^{-\alpha y} \dd y = 0.
\] 
Since for $\alpha > 3/4$, $s_{3/4}(k_n) = \log(k_n)^{-1} s_\alpha(k_n) = \smallO{s_\alpha(k_n)}$ it suffices to prove the following two cases:
\begin{enumerate}
\item if $1/2 < \alpha \leq 3/4$, then
\[
	\lim_{n\to \infty} k_n^{6\alpha - 3} \, \int_{\Kcal_{C}(k_n)} 
		\Exp{\left|\ind{\mathrm{deg}_{n}(y) = k_n} T_{\Po}(y)
		- \ind{\mathrm{deg}_{\infty}(y) = k_n} P(y)\right|} e^{-\alpha y} \dd y = 0,
\]
\item if $3/4 < \alpha$, then
\[ 
	\lim_{n\to \infty} k_n^{2\alpha} \, \int_{\Kcal_{C}(k_n)} 
		\Exp{\left|\ind{\mathrm{deg}_{n}(y) = k_n} T_{\Po}(y)
		- \ind{\mathrm{deg}_{\infty}(y) = k_n} P(y)\right|} e^{-\alpha y} \dd y = 0.
\]
\end{enumerate}
We shall proceed by expanding the integrand and analyzing the individual terms. With a slight abuse of notation we shall write $y$ instead of $(0,y)$ in expression such as $\BallHyp{y}$. In addition we write $D_{\Po}(y,k_n;\Pcal)$ for the indicator which is equal to 1 if and only if $\BallHyp{y}$ contains $k_n$ points from $\Pcal \setminus \{(0,y)\}$. We define $D_{\text{box}}(y,k_n;\Pcal)$ analogously for the ball $\BallPon{y}$. It is important to note that for any $p^\prime \in \Rcal$ it holds that $p^\prime \in \BallPon{y} \iff p^\prime \in \BallPo{y}$.




We need to split the integrand over several terms and then analyze each of these separately. Applying the Campbel-Mecke formula yields
\begin{align*} 
 &\hspace{-30pt}\Exp{ \left| \ind{\mathrm{deg}_{n}(y) = k_n} P_{\Po}(y)
        - \ind{\mathrm{deg}_{\infty}(y) = k_n}  P(y)
        \right|}\leq \\
 & {\mathbb E} \left[ \sum_{(p_1,p_2) \in \Pcal \setminus \{(0,y)\}}^{\not =} 
  \left| D_{\Po}(y,k_n-2; \Pcal \setminus \{p_1,p_2 \}) T_{\Po}(y,p_1,p_2) \right. \right.\\
  & \hspace{5cm} 
\left. \left. - D_{\text{box}} (y,k_n-2;\Pcal \setminus \{p_1,p_2\}) \widetilde{T}_{\text{box}}(y,p_1,p_2)
   \right| \vphantom{\sum_{p_1,p_2 \in \Pcal \setminus \{(0,y)\}}^{\not =}}\right],
\end{align*}
where the sum ranges over all distinct pairs of points in $\Pcal \setminus \{ (0,y)\}$. In what follows, we will set $\BallSym{p'} = \BallHyp{p'} \bigtriangleup (\BallPo{p'} \cap \Rcal)$ and $\BallInter{p'} =\BallHyp{p'} \cap \BallPon{p'}$ and observe that $\BallInter{y} = \BallHyp{y} \cap \BallPo{y}$. We will now bound the sum that is inside the expectation.
We will split these summands classes of all combinations of $p_1, p_2\in \Pcal \setminus \{(0,y)\}$ for which only one of the two terms of this difference is non-zero. Clearly, for this we need that either $p_1 \in \BallInter{y}$ and $p_2 \in \BallSym{p_1}$ or $p_1 \in \BallSym{y}$ and $p_2 \in \BallInter{p_1}$. We will consider the following four classes:
\begin{enumerate}
\item$p_1 \in \BallInter{y}$ and $p_2 \in \BallSym{p_1}$
\begin{enumerate}
\item $y_1,y_2 < (1-\eps ) R_n \wedge (R_n-y)$
\item $y_1 \ge (1-\eps ) R_n \wedge (R_n-y)$
\end{enumerate}
\item$p_1 \in \BallHyp{y} \setminus \BallPo{y}$ with $y_1 < K$ and $p_2 \in \BallInter{y}$.
\item $p_1 \in \BallSym{y}$ with $y_1 \ge K$ and $p_2 \in \BallInter{y}$.
\end{enumerate}
Observe that when $y_1 < (1-\eps ) R_n \wedge (R_n-y)$ and $y_2 \ge (1-\eps ) R_n \wedge (R_n-y)$ it follow from Corollary~\ref{cor:balls_inclusion} that $p_2 \in \BallInter{p_1}$ and thus we do not have to consider this case when $p_1 \in \BallInter{y}$ and $p_2 \in \BallSym{p_1}$. Similarly, when $y_1 \ge K$ $p_1 \in \BallSym{y}$ Corollary~\ref{cor:balls_inclusion} implies that $p_1 \in \BallHyp{y} \setminus \BallPo{y}$ which explains the setting of case 2.

%\begin{enumerate} 
%\item both $p_1$ and $p_2$ have $y_1,y_2 < (1-\eps ) R_n \wedge (R_n-y)$ and 
%\begin{enumerate}
%\item $p_1$ is in $\BallInter{y}$ but $p_2 \in \BallHyp{p_1} \setminus \BallPo{p_1}$ 
%and $\BallHyp{y}$ contains exactly $k_n-2$ or $k_n-1$ other points (depending on whether 
%$p_2 \in \BallHyp{y}$ or not).
%\item $p_1$ is in $\BallInter{y}$ but $p_2 \in (\BallPo{p_1} \cap \Rcal) \setminus \BallHyp{p_1}$ 
%and $\BallPo{y}$ contains exactly $k_n-2$ or $k_n-1$ other points (depending on whether 
%$p_2 \in \BallPo{y}$ or not).
%\end{enumerate}
%\item the above cases but with $y_1 \geq (1-\eps) R_n \wedge (R_n -y)$. 
%\item $y_1 \geq K$ and $p_1 \in \BallHyp{y} \setminus 
%\BallPon{y}$ and $p_2 \in \BallInter{y}$ - here we use 
%Corollary~\ref{cor:balls_inclusion} which implies that if $p_1 \in \BallSym{y}$ and $y_1 \geq K$, then in fact $p_1 \in \BallHyp{y} \setminus \BallPo{y}$. 
%\item $y_1 < K$ and $p_1 \in \BallSym{y}$ and $p_2 \in \BallInter{y}$. 
%%\item $p_1$ and $p_2$ are such that $\Delta_\H ((0,y),p_1,p_2)=\Delta_\Pcal ((0,y),p_1,p_2)=1$. 
%\end{enumerate}
We can now bound sum by the following expression:
\begin{align} 
	&\sum_{(p_1,p_2) \in \Pcal \setminus \{(0,y)\}}^{\not =} 
	  \left| D_{\Po}(y,k_n-2; \Pcal \setminus \{p_1,p_2 \}) T_{\Po} (y,p_1,p_2) \right. \notag \\
	& \hspace{5cm} \left. - D_{\text{box}} (y,k_n-2;\Pcal \setminus \{p_1,p_2\}) \widetilde{T}_{\text{box}} (y,p_1,p_2)
	   \right|  \notag \\
	&\le \hspace{-15pt} \sum_{p_1,p_2\in \Pcal \setminus \{(0,y)\} \atop  y_1, y_2 < (1-\eps)R_n \wedge (R_n-y)} 
		\hspace{-10pt} \ind{p_1 \in \BallInter{y}} \cdot \ind{p_2 \in \BallSym{p_1}} 
		\, D_{\Po} (y,k_n-2;\Pcal \setminus \{ p_1,p_2\}) \label{eq:sum_1}\\
	&\le \hspace{-15pt} \sum_{p_1,p_2\in \Pcal \setminus \{(0,y)\} \atop  y_1, y_2 < (1-\eps)R_n \wedge (R_n-y)} 
		\hspace{-10pt} \ind{p_1 \in \BallInter{y}} \cdot \ind{p_2 \in \BallSym{p_1}} 
		\, D_{\text{box}} (y,k_n-2;\Pcal \setminus \{ p_1,p_2\}) \label{eq:sum_2}\\
	&\hspace{10pt}+  \hspace{-15pt} \sum_{p_1,p_2 \ \in \Pcal \setminus \{(0,y) \} 
		\atop y_1 \geq (1-\eps) R_n \wedge (R_n-y)} \hspace{-10pt}
		\ind{p_1 \in \BallInter{y}} \cdot \ind{p_2 \in \BallSym{p_1}\cap \BallHyp{y}} 
		\, D_{\Po} (y,k_n-2;\Pcal \setminus \{ p_1,p_2\}) \label{eq:sum_3} \\
	&\hspace{10pt}+ \hspace{-15pt} \sum_{p_1,p_2 \ \in \Pcal \setminus \{(0,y) \} 
		\atop y_1 \geq (1-\eps) R_n \wedge (R_n-y)} \hspace{-10pt}
		\ind{p_1 \in \BallInter{y}} \cdot \ind{p_2 \in \BallSym{p_1}\cap \BallHyp{y}} 
		\, D_{\text{box}} (y,k_n-2;\Pcal \setminus \{ p_1,p_2\}) \label{eq:sum_4} \\
	&\hspace{10pt}+ \hspace{-10pt} \sum_{p_1,p_2 \in\Pcal \setminus \{(0,y)\} 
		\atop y(p_1) \geq K} \hspace{-10pt} \ind{p_1\in \BallHyp{y}\setminus \BallPo{y}} \ind{p_2\in \BallHyp{y}\cap \BallPo{y}} 
		\, D_{\Po} (y,k_n-2;\Pcal \setminus \{ p_1,p_2\}) \label{eq:sum_5}\\
	&\hspace{10pt}+ \hspace{-10pt} \sum_{p_1,p_2 \in\Pcal \setminus \{(0,y)\} 
			\atop y(p_1) \geq K} \hspace{-10pt} \ind{p_1\in \BallHyp{y}\setminus \BallPo{y}} \ind{p_2\in \BallHyp{y}\cap \BallPo{y}} 
			\, D_{\text{box}} (y,k_n-2;\Pcal \setminus \{ p_1,p_2\}) \label{eq:sum_5b}\\
	&\hspace{10pt}+ \hspace{-10pt} \sum_{p_1,p_2 \in\Pcal \setminus \{(0,y)\} \atop \ y(p_1) < K}
		\ind{p_1\in \BallSym{y}} \ind{p_2\in \BallHyp{y}\cap \BallPo{y}} \label{eq:sum_6}
%& +  \sum_{p_1,p_2 \in \Pcal \setminus \{(0,y)\}}^{\not =} \ind{p_1,p_2 \in \BallHyp{(0,y)}\cup \BallPo{(0,y)}} \cdot 
%D_{\Po}(y,k_n-2; \Pcal \setminus \{ p_1,p_2\})  \cdot 
%\left| \expH^{-1} - \expP^{-1} \right|.  \label{eq:sum_7}
\end{align}
In the following paragraphs we will give upper bounds on the expected values of each one of these partial sums. 

\paragraph{The sums~\eqref{eq:sum_1} and~\eqref{eq:sum_2}}

We will analyze~\eqref{eq:sum_1}. The analysis of the other sum~\eqref{eq:sum_2} is similar.
Note first that for any two points $p_1,p_2$ the following holds: $p_1 \in \BallHyp{y}$ and $p_2 \in \BallSym{p_1}\cap \BallHyp{y}$, then $p_2 \in \BallHyp{y}$ and $p_1 \in \BallSym{p_2}\cap \BallHyp{y}$.
Using this symmetry, it suffices to consider distinct pairs $(p_1,p_2) \in \Pcal \setminus \{(0,y)\}$ with $0\leq y_2 \leq y_1 \leq R- y$. Let $\dom{}$ denote the set of these pairs. 

We are going to consider several sub-cases and, thereby, split the domain $\dom{}$ into the corresponding sub-domains. 
Let $\omega =\omega (n) \to \infty$ as $n\to \infty$ be a slowly growing function and set $y_\omega := y +\omega$. 
We let 
\begin{align*}
	\dom{1} &= \{(p_1,p_2) \in \dom{} \ : \ y \leq y_1 \leq R_n/2, \, y_\omega \leq y_2 \leq y_1 \},\\
	\dom{2} &= \{(p_1,p_2) \in \dom{} \ : \ y_1 \leq R_n/2, \, y_2 \leq y_\omega \} \text{ and }\\
	\dom{3} &=  \{(p_1,p_2) \in \dom{} \ : R_n/2 < y_1 \leq R_n -y, \, y_2 \leq y_1 \}.
\end{align*} 
Note that $\dom{} \subseteq \dom{1} \cup \dom{2}\cup \dom{3}$.
Hence, we can write 
\begin{equation} \label{eq:1sum-rewriting}
\begin{split} 
	&\Exp{\sum_{p_1, p_2\in \Pcal \setminus \{(0,y)\} 
	\atop y_1, y_2 \leq (1-\eps) R_n\wedge (R_n-y)} \hspace{-10pt} \ind{p_1 \in \BallHyp{y}} 
	\ind{p_2 \in \BallSym{p_1}\cap \BallHyp{y}} 
	\, D_{\Po} (y,k_n-2;\Pcal \setminus \{p_1,p_2\})} \\ 
	&\le \sum_{i=1}^3 \Exp{\sum_{(p_1, p_2)\in \dom{i}} \ind{p_1 \in \BallHyp{y}} 
	\ind{p_2 \in \BallSym{p_1}\cap \BallHyp{y}} \cdot D_{\Po} (y,k_n-2;\Pcal \setminus \{p_1,p_2\})}.
\end{split}
\end{equation}
We bound each one of the above three summands as follows:  
\begin{equation} \label{eq:term1}
\begin{split}
	&\Exp{\sum_{(p_1, p_2)\in \dom{1}} \ind{p_1 \in \BallHyp{y}} \cdot \ind{p_2 \in \BallSym{p_1}\cap \BallHyp{y}} 
		\, D_{\Po} (y,k_n-2;\Pcal \setminus \{p_1,p_2\})} \\
	&\le \Exp{\sum_{(p_1, p_2)\in \dom{1}} \ind{p_1 \in \BallHyp{y}} \cdot \ind{p_2 \in  \BallHyp{y}} 
		\, D_{\Po} (y,k_n-2;\Pcal \setminus \{p_1,p_2\})} := \mathcal{I}_n^{(1)}(y),
\end{split}
\end{equation}

\begin{equation} \label{eq:term2}
\begin{split}
	&\Exp{\sum_{(p_1, p_2)\in \dom{2}} \ind{p_1 \in \BallHyp{y}} \cdot \ind{p_2 \in \BallSym{p_1}\cap \BallHyp{y}} 
		\, D_{\Po} (y,k_n-2;\Pcal \setminus \{p_1,p_2\})} \\
	&\le \Exp{\sum_{(p_1, p_2)\in \dom{2}} \ind{p_1 \in \BallHyp{y}} \cdot \ind{p_2 \in \BallSym{p_1}} 
		\, D_{\Po} (y,k_n-2;\Pcal \setminus \{p_1,p_2\})} := \mathcal{I}_n^{(2)}(y)
\end{split}
\end{equation}
and 
\begin{equation}\label{eq:term3}
\begin{split}
	&\Exp{\sum_{(p_1, p_2)\in \dom{3}} \ind{p_1 \in \BallHyp{y}} \cdot \ind{p_2 \in \BallSym{p_1}\cap \BallHyp{y}} 
		\, D_{\Po} (y,k_n-2;\Pcal \setminus \{p_1,p_2\})} \\
	&\le \Exp{\sum_{(p_1, p_2)\in \dom{3}} \ind{p_1 \in \BallHyp{y}} \cdot \ind{p_2 \in \BallHyp{y}} 
		\, D_{\Po} (y,k_n-2;\Pcal \setminus \{p_1,p_2\})} := \mathcal{I}_n^{(3)}(y).
\end{split}
\end{equation}
We will bound each term using the Campbell-Mecke formula and show for $i = 1,2,3$ that for $1/2 < \alpha < 3/4$
\begin{equation}
	\lim_{n \to \infty} k_n^{6\alpha -3} 
		\int_{\Kcal_{C}(k_n)} \mathcal{I}_n^{(i)}(y) e^{-\alpha} \dd y = 0,
\end{equation}
and for $\alpha \ge 3/4$
\begin{equation}
	\lim_{n \to \infty} k_n^{2\alpha}
		\int_{\Kcal_{C}(k_n)} \mathcal{I}_n^{(i)}(y) e^{-\alpha} \dd y = 0.
\end{equation}

For the first term~\eqref{eq:term1}, we note that
\[ 
\Exp{D_{\Po}(y, k_n-2;\Pcal \setminus \{p_1, p_2\}} =\rho_{\Po}(y,k_n-2).
\] 
and hence $\mathcal{I}_n^{(1)}(y)$ becomes
\begin{equation} \label{eq:1sum-expansion} 
\begin{split}
	\rho_{\Po}(y,k_n-2) \int_{-I_n}^{I_n} \int_y^{R_n/2}\int_{-I_n}^{I_n} \int_{y_\omega}^{y_1}
  	\ind{p_1 \in \BallInter{y}} \, \ind{p_2 \in \BallHyp{y}} 
  	e^{-\alpha (y_1 + y_2)} \dd y_2 \dd x_2 \dd y_1 \dd x_1.
\end{split}
\end{equation}

Next, Lemma~\ref{lem:asymptotics_Omega_hyperbolic} implies that for $y'\leq R_n -y$, we have that if $(x',y') \in \BallHyp{y}$, then $|x'| < (1+ K) e^{y/2 + y'/2}$, where $K >0$ is as in Lemma~\ref{lem:asymptotics_Omega_hyperbolic}. Using these observations, we obtain: 
\begin{align*}
	&\hspace{-30pt} \Exp{\sum_{p_1, p_2\in \dom{1}} \ind{p_1 \in \BallInter{(0,y)}} \cdot \ind{p_2 \in \BallHyp{y}} \cdot 
	D_{\Po} (y,k_n-2;\Pcal \setminus \{p_1, p_2\})} \\
	&= \rho_{\Po}(y,k_n-2)
	e^{y}\int_y^{R_n/2} e^{y_1/2}\int_{y_\omega}^{y_1} e^{y_2/2} e^{-\alpha y_2} \cdot e^{-\alpha y_1} dy_2 dy_1.
\end{align*}

Now, the double integral becomes
\begin{equation}
\begin{split}
& \int_y^{R_n/2} e^{y_1/2}\int_{y_\omega}^{y_1} e^{y_2/2} e^{-\alpha y_2} \cdot e^{-\alpha y_1} dy_2 dy_1 = \\
&  O(1) \cdot  \int_y^{R_n/2} e^{y_1/2 - \alpha y_1} \cdot 
e^{(1/2 - \alpha) y_\omega} dy_1 \\
& =O(1) \cdot e^{(1/2 - \alpha) y_\omega} \cdot \int_y^{R_n/2} e^{y_1/2 - \alpha y_1} d y_1 \\
& =O(1) \cdot e^{(1/2 - \alpha) y_\omega + (1/2 - \alpha) y} \\ 
& \ll e^{(1 - 2\alpha) y},
\end{split}
\end{equation}
since $y_\omega = y + \omega$ and $\omega \to \infty$. 
We then deduce that 
\begin{equation}
\begin{split}
&\hspace{-50pt}\Exp{\sum_{p_1, p_2\in \dom{1}} \ind{p_1 \in \BallInter{(0,y)}} \cdot \ind{p_2 \in \BallHyp{y}} \cdot 
D_{\Po} (y,k_n-2;\Pcal \setminus \{p_1, p_2\})}\\
& \ll \rho_{\Po}(y,k_n-2) e^{(1-2\alpha)y}.
\end{split}
\end{equation}
We now integrate this with respect to $y$ and determine its contribution to~\eqref{eq:expectation_total} is 
\begin{align*} 
	& \int_{\Kcal_{C}(k_n)} \rho_{\Po}(y,k_n-2) e^{(1-2\alpha)y} e^{-\alpha y} \dd y \dd x \\
	&= \bigO{k_n^{-6\alpha + 1}}
\end{align*}
where we used Lemma~\ref{lem:gamma_approx} with $t = 1 - 2\alpha$.

Since $1 - 6\alpha + \min\{6\alpha - 3, 2\alpha\} < 0$ for all $\alpha > 1/2$ we deduce that for $1/2 < \alpha < 3/4$
\[
	\lim_{n \to \infty} k_n^{6\alpha - 3} 
	\int_{\Kcal_{C}(k_n)} \mathcal{I}_n^{(1)}(y) e^{-\alpha y} \dd y = 0,
\]
while for $\alpha \ge 3/4$
\[
	\lim_{n \to \infty} k_n^{2\alpha}  
		\int_{\Kcal_{C}(k_n)} \mathcal{I}_n^{(1)}(y) e^{-\alpha y} \dd y = 0.
\]

We will now bound the term in~\eqref{eq:term2}. Using similar observations as for the previous term we get that $\mathcal{I}_n^{(2)}(y)$ equals
\[
	 \rho_{\Po}(y,k_n-2) \int_{-I_n}^{I_n} \int_0^{R_n/2} \hspace{-2pt} \int_{-I_n}^{I_n} \int_0^{y_\omega}
	 \ind{p_1 \in \BallHyp{y}} \, \ind{p_2 \in \BallSym{(0,y)}} e^{-\alpha(y_1 + y_2)} \dd y_2 
	 \dd x_2 \dd y_1 \dd x_1.
\]

Now, Lemma~\ref{lem:asymptotics_Omega_hyperbolic} implies that for 
$y_2 \leq R_n -y_1$, we have that if 
$(x_2,y_2) \in \BallSym{(x_1,y_1)}$, then $x_2$ lies in an interval of length 
$Ke^{3 y/2 + 3y'/2 - R_n}$, where $K >0$ is again the constant in Lemma~\ref{lem:asymptotics_Omega_hyperbolic}. 
%For $y> R_n - y$, we will simply take $|x'| < I_n$. 
Using these observations we obtain: 
\begin{equation} \label{eq:term2_intermediate}
	\mathcal{I}_n^{(2)}(y) = \rho_{\Po}(y,k_n-2) e^{y/2}\int_0^{R_n/2} e^{y_1/2 + 3y_1/2}
	\int_0^{y_\omega} e^{3y_2/2 - R_n} e^{-\alpha y_2} \cdot e^{-\alpha y_1} \dd y_2 \dd y_1. 
\end{equation} 
Now the latter integral is
\begin{align*}
	&\hspace{-30pt} e^{-R_n}  \left(\int_0^{R_n/2} e^{(2-\alpha )y_1} \dd y_1\right) 
		\left( \int_0^{y} e^{(3/2 - \alpha) y_2} \dd y_2 \right)\\
	&= \bigO{1} e^{-R_n} \left(\begin{cases}
		e^{(1-\alpha/2)R_n} &\mbox{if } \frac{1}{2} < \alpha < 2\\
		R_n &\mbox{if } \alpha \ge 2
	\end{cases}\right)
	\left(\begin{cases}
		e^{(3/2 - \alpha)y} &\mbox{if } \frac{1}{2} < \alpha < \frac{3}{2}\\
		y &\mbox{if } \alpha \ge \frac{3}{2}
	\end{cases}\right)\\
	&= \bigO{1} \begin{cases}
		e^{-\frac{\alpha}{2}R_n} e^{(3/2 - \alpha)y} &\mbox{if } \frac{1}{2} < \alpha < \frac{3}{2}\\
		y e^{\frac{\alpha}{2} R_n} &\mbox{if } \frac{3}{2} \le \alpha < 2\\
		y R_n e^{-R_n} &\mbox{if } \alpha \ge 2.
	\end{cases} 
\end{align*}
Since $y \le R_n = \bigO{\log(n)}$ we conclude that on $\Kcal_{C}(k_n)$
\[
	\mathcal{I}_n^{(2)}(y) = \bigO{1} \rho_{\Po}(y,k_n-2) \begin{cases}
				n^{-\alpha} k_n^{3 - 2\alpha} &\mbox{if } \frac{1}{2} < \alpha < \frac{3}{2}\\
				n^{-\alpha} \log(n) &\mbox{if } \frac{3}{2} \le \alpha < 2\\
				n^{-2} \log(n)^{2} &\mbox{if } \alpha \ge 2,
		\end{cases}	
\]
and hence
\begin{align*}
	\int_{\Kcal_{C}(k_n)} \mathcal{I}_n^{(2)}(y) e^{-\alpha y} \dd y
	&= \bigO{1} k_n^{-(2\alpha + 1)}\begin{cases}
				n^{-\alpha} k_n^{3 - 2\alpha} &\mbox{if } \frac{1}{2} < \alpha < \frac{3}{2}\\
				n^{-\alpha} \log(n) &\mbox{if } \frac{3}{2} \le \alpha < 2\\
				n^{-2} \log(n)^{2} &\mbox{if } \alpha \ge 2,
		\end{cases}	\\
	&= \bigO{1} \begin{cases}
				n^{-\alpha} k_n^{2 - 4\alpha} &\mbox{if } \frac{1}{2} < \alpha < \frac{3}{2}\\
				n^{-\alpha} \log(n)k_n^{-(2\alpha + 1)} &\mbox{if } \frac{3}{2} \le \alpha < 2\\
				n^{-2} \log(n)^{2} k_n^{-(2\alpha + 1)} &\mbox{if } \alpha \ge 2.
		\end{cases}
\end{align*}

Now for $1/2 < \alpha < 3/4$ it holds that $4\alpha^2 - \alpha + 1 > 0$. Hence since $k_n = \bigO{n^{\frac{1}{2\alpha + 1}}}$, we have
\[
	k_n^{6\alpha - 3} n^{-\alpha} k_n^{2 - 4\alpha} = n^{-\alpha} k_n^{2\alpha - 1} = \bigO{n^{-\alpha + \frac{2\alpha - 1}{2\alpha + 1}}} = \bigO{k_n^{-\frac{4\alpha^2 - \alpha + 1}{2\alpha + 1}}} = \smallO{1},
\]
from which we deduce that
\[
	\lim_{n \to \infty} k_n^{6\alpha - 3} \int_{\Kcal_{C}(k_n)} \mathcal{I}_n^{(2)}(y)
	e^{-\alpha y} \dd y = 0.
\]
For $\alpha \ge 3/4$ we have that both $n^{-\alpha} \log(n) k_n^{-1}$ and $n^{-2} \log(n)^2 k_n^{-1}$ converge to zero as $n \to \infty$ and hence in this case
\[
	\lim_{n \to \infty} k_n^{2\alpha} \int_{\Kcal_{C}(k_n)} \mathcal{I}_n^{(2)}(y)
	e^{-\alpha y} \dd y = 0.
\]

We will now consider the term in~\eqref{eq:term3}. 
Recall that $\dom{3}$ consists of all pairs $(p_1, p_2 ) \in \dom{}$ such that $R_n/2 < y_1 \leq (1-\eps)R_n \wedge (R_n-y)$ and $y_2 \leq y_\omega$ with the property that 
$p_1 \in \BallHyp{y}$ and $p_2 \in \BallSym{p_1} \cap \BallHyp{y}$.    
So, in particular, $p_2 \in (\BallHyp{p_1} \cup \BallPo{p_1}) \cap \BallHyp{y}$.

We will consider this intersection more closely. We use Lemma~\ref{lem:asymptotics_Omega_hyperbolic} to define a ball around $p_1$ that contains both 
$\BallHyp{p_1}$ and $\BallPo{p_1}$.
For $K > 0$, we define, for any point $p_1=(x_1,y_1) \in \R \times \R_+$,
\begin{equation}\label{eq:def_fatball}
	\FatBallHyp{p_1} : = \{ (x',y') \ : \  y' < R_n - y_1, \ | x_1 - x'| < (1+K) e^{\frac{1}{2} (y_1 + y')}  \}.
\end{equation}
%Note that $\Delta (r(p),r') = \frac12 e^{R/2} \theta_R (p)$.
It is an implication of Lemma~\ref{lem:asymptotics_Omega_hyperbolic}  that 
\begin{equation*} %\label{eq:ball_inclusion} 
(\BallHyp{p_1} \cup \BallPo{p_1}) \cap \Rcal([0, R_n - y_1]) \subseteq \FatBallHyp{p_1}
\end{equation*}
Therefore, any point $p_2 = (x_2,y_2) \in \BallSym{p_1} \cap \BallHyp{y}$ with 
$y_2 \leq R-y_1$ must belong to $\FatBallHyp{p_1} \cap \FatBallHyp{y}$.

We will use this in order to derive a lower bound on $y_2$ as a function of $x_1, y_1$. 
Let us suppose without loss of generality that $x_1 < 0$. 
The left boundary of $\FatBallHyp{(0,y)}$ is given by the equation 
$x^\prime = (1+K)e^{\frac{1}{2} (y + y^\prime)}$ whereas the right boundary of $\FatBallHyp{p_1}$ is given by the curve having equation $x^\prime = x_1 + (1+ K)e^{\frac{1}{2} (y_1 + y^\prime)}.$
The equation that determines the intersection point $(\hat{x},\hat{y})$ of these curves  is
\[
	x_1 + (1+K)e^{(y_1 + \hat{y})/2}= (1+K) e^{(y + \hat{y})/2}.
\]
We can solve the above for $\hat{y}$  
\begin{equation*} 
\begin{split}
|x_1| &=(1+K) e^{\hat{y}/2} \left( e^{y_1/2} + e^{y/2} \right).
\end{split}
\end{equation*}
But $y_1 > R_n/2$ and since $y \in \Kcal_{C}(k_n)$, it follows that $y \le \bigO{1}(1+\eps) R_n /(2\alpha +1)$. So if $\eps$ is small enough depending on $\alpha$, we have 
$$ |x_1| =(1+K) e^{\hat{y}/2} \left( e^{y_1/2} + e^{y/2} \right) = (1+K+o(1))e^{\hat{y}/2 + y_1/2}. $$
Let $c_K$ denote the multiplicative term $1+ K+o(1)$, which appears in the above.
The above yields
\begin{equation} \label{eq:to_use_I1}
\hat{y}= \left(2 \log(|x_1|e^{-y_1/2}) - \log c_K \right) \vee 0 := \hat{y}(x_1,y_1). 
\end{equation}
In particular, note that $\hat{y} = 0$ if and only if $|x_1| \leq c_K e^{y_1/2}$.  
Moreover, since $p_1 \in \BallHyp{y}$ and $x_1 \leq R_n - y$, we also have that 
$|x_1| \leq e^{(y+y_1)/2} (1+o(1))$. This upper bound on $|x_1|$ together with~\eqref{eq:to_use_I1}, imply that for $n$ sufficiently large, we have $\hat{y} \leq y$. This observation will be used below, where 
we integrate over $y_2$, thus ensuring that the integrals are non-zero. 

We conclude that 
\begin{equation*}\label{eq:intersex_approx}
	p^\prime \in \FatBallHyp{y}\cap \FatBallHyp{(x_1,y_1)} \Rightarrow y^\prime \ge \hat{y}(x_1,y_1),
\end{equation*}
which implies
\begin{equation} \label{eq:symdiff_loc}
\begin{split} 
 \ind{p_2 \in \BallSym{p_1}\cap \BallHyp{y}} \leq \ind{y_2 \geq \hat{y}(x_1,y_1), p_2 \in \FatBallHyp{(0,y)}}.
\end{split}
\end{equation}
If we integrate this over $x_2, y_2$ we get 
\begin{align*}
\int_{-I_n}^{I_n} \int_{0}^{y_\omega}  \ind{p_2 \in \BallSym{p_1}\cap \BallHyp{y}}  
e^{-\alpha y_2} dy_2 dx_2
&  \leq 
\int_{-I_n}^{I_n} \int_{0}^{y_\omega}  \ind{y_2 \geq \hat{y}(x_1,y_1), p_2 \in \FatBallHyp{y}}  
e^{-\alpha y_2} dy_2 dx_2 \\
&\leq (1+K) \cdot e^{y/2} \int_{\hat{y}(x_1,y_1)}^{y_\omega} e^{y_2/2 - \alpha y_2} dy_2 \\
&=O(1) \cdot e^{y/2 + (1/2 -\alpha) \hat{y}(x_1,y_1)}.
\end{align*}
Note also that 
\[
	\Exp{D_{\Po} (y,k_n-2;\Pcal \setminus \{p_1, p_2\})} = \rho_{\Po}(y,k_n-2),
\]
uniformly over all $(p_1, p_2) \in \dom{3}$. 

So the Campbell-Mecke formula yields that $\mathcal{I}_n^{(3)}(y)$ equals: 
\begin{align*}
 	&O(1) \rho_{\Po}(y,k_n-2) \, e^{y/2} \int_{-I_n}^{I_n} \int_{R_n/2}^{(R_n - y) \wedge (1-\eps)R_n}  
		\ind{p_1 \in \BallHyp{y}} e^{(1/2 -\alpha) \hat{y}(x_1,y_1) - \alpha y_1} dy_1dx_1 \\
	&= O(1) \rho_{\Po}(y,k_n-2) \, e^{y/2} \int_{-I_n}^{I_n} \int_{R_n/2}^{(R_n - y) \wedge (1-\eps)R_n}  
		\ind{p_1 \in \FatBallHyp{y}} e^{(1/2 -\alpha) \hat{y}(x_1,y_1) - \alpha y_1} dy_1dx_1.
\end{align*}
Due to the symmetry of $\FatBallHyp{y}$, the integration over $x_1$ is: 
\[
	O(1) \cdot e^{y/2} \cdot \int_0^{(1+K)e^{y/2 + y_1/2}} e^{\hat{y}(x_1,y_1) (1/2 -\alpha)} dx_1
\]
We will split this integral into two parts according to the value of $\hat{y}(x_1,y_1)$:
\[
\int_0^{(1+K) e^{y/2 + y_1/2}} e^{\hat{y}(x_1,y_1) (1/2 -\alpha)} dx_1 = 
\int_{c_K e^{y_1/2}}^{(1+K)e^{y/2 + y_1/2}} e^{\hat{y}(x_1,y_1) (1/2 -\alpha)} dx_1 + \int_0^{c_K e^{y_1/2}} dx_1.
\]
The first integral becomes: 
\begin{align*}
&\int_{c_K e^{y_1/2}}^{(1+K)e^{y/2 + y_1/2}} e^{\hat{y}(x_1,y_1) (1/2 -\alpha)} dx_1  = 
\int_{c_K e^{y_1/2}}^{(1+K)e^{y/2 + y_1/2}} e^{\hat{y}(x_1,y_1)/2 (1 -2\alpha)} dx_1  \\
&= O(1)\cdot \int_{c_K e^{y_1/2}}^{(1+K)e^{y/2 + y_1/2}} x_1^{1 -2\alpha} 
e^{-\frac{y_1}{2} (1-2\alpha)} dx_1 \\
&= O(1) \cdot e^{-y_1/2 + \alpha y_1} \cdot e^{\frac{(y+y_1)}{2} 2(1-\alpha)} \\
&=O(1) \cdot e^{y_1/2 +y(1-\alpha)}.  
\end{align*}
The second integral trivially gives: 
\[
	\int_0^{c_K e^{y_1/2}} dx_1 = O(1) \cdot e^{y_1/2} = O(1) \cdot e^{y_1/2 +y(1-\alpha)}.
\]
We conclude that 
\[
	e^{y/2} \cdot \int_0^{(1+K)e^{y/2 + y_1/2}} e^{\hat{y}(x_1,y_1) (1/2 -\alpha)} dx_1 = 
	O(1) \cdot e^{y_1/2 +y(3/2-\alpha)}.
\]
Now, we integrate this with respect to $y_1$ and get 
\[
	e^{y(3/2 -\alpha)} \int_{R_n/2}^{R_n-y} e^{(1/2-\alpha)y_1} dy_1 =  O(1) \cdot e^{y(3/2 -\alpha)} 
	e^{(1/2 -\alpha) R_n/2} = O(1) \cdot n^{1/2 -\alpha} \cdot e^{y(3/2 - \alpha)},
\]
from which we deduce
\begin{equation} \label{eq:term3_intermediate}
	\mathcal{I}_n^{(3)}(y) = 
	O(1) \cdot n^{1/2 -\alpha} e^{y(3/2 - \alpha)} \, \rho_{\Po}(y,k_n-2). 
\end{equation}
We now apply Lemma~\ref{lem:gamma_approx} with $t = \frac{3}{2} - \alpha$ and get 
\begin{align*}
	\int_{\Kcal_{C}(k_n)} \mathcal{I}_n^{(3)}(y) e^{-\alpha y} \dd y
	&= \bigO{1} n^{-(\alpha - \frac{1}{2})} \int_{\Kcal_{C}(k_n)} e^{(3/2 - \alpha)y} \rho_{\Po}(y,k_n-2) e^{-\alpha y} 
		\dd y\\
	&= \bigO{n^{-(\alpha - \frac{1}{2})} k_n^{2 -4\alpha}}.
\end{align*}

Since for $\alpha > 1/2$, $k_n = \bigO{n^{\frac{1}{2\alpha +1}}} = \smallO{n^{1/2}}$ we have that $k_n^{6\alpha - 3} k_n^{2 - 4\alpha} n^{-(\alpha - 1/2)} = \smallO{1}$ and hence for $1/2 < \alpha < 3/4$.
\[
	\lim_{n \to \infty} k_n^{6\alpha - 3} \int_{\Kcal_{C}(k_n)} \mathcal{I}_n^{(3)}(y) e^{-\alpha y} \dd x \dd y = 0,
\]
For $\alpha \ge 3/4$ we observe that $2\alpha^2 + 2\alpha - 5/2 > 0$. Hence,
\[
	k_n^{2\alpha} n^{-(\alpha - \frac{1}{2})} k_n^{2 -4\alpha} 
	= \bigO{n^{-(\alpha - 1/2)} n^{\frac{2 - 2\alpha}{2\alpha + 1}}}
 	= \bigO{n^{- \frac{2\alpha^2 + 2\alpha - 5/2}{2\alpha + 1}}} = \smallO{1}.
\]
and we get for $\alpha \ge 3/4$
\[
	\lim_{n \to \infty} k_n^{2\alpha} \int_{\Kcal_{C}(k_n)} \mathcal{I}_n^{(3)}(y) e^{-\alpha y} \dd x \dd y = 0.
\]



\paragraph{The sums~\eqref{eq:sum_3} and~\eqref{eq:sum_4}}
Again, we will only consider~\eqref{eq:sum_3} since the analysis for the other term is similar. Recall that in this case, we consider pairs $(p_1,p_2)$, with $p_1 = (x_1,y_1)$ satisfying 
$y_1 \geq (R_n - y) \wedge (1-\eps)R_n$, and $p_1 \in \BallHyp{y}$, $p_2 \in \BallSym{p_1} \cap \BallHyp{y}$. 
We split this into three sub-domains:  i) $y_2 \geq R_n - y$; ii) $R_n -y_1 \leq y_2 \leq R_n -y$ and iii) $y_2 < R_n - y_1$. Similar to the analysis above we define
\begin{align*}
	\dom{1} &:= \{(p_1, p_2) \ : \  p_1,p_2 \ \in \Pcal \setminus \{(0,y) \}, \ y_1 \geq (1-\eps) R_n \wedge (R_n-y), 
		\ R_n -y \leq y_2 \leq R_n \}\\
	\dom{2} &:= \{(p_1, p_2) \ : \  p_1,p_2 \ \in \Pcal \setminus \{(0,y) \}, \ y_1 \geq (1-\eps) R_n \wedge (R_n-y), 
		\ R_n -y_1 \leq y_2 \leq R_n - y \}\\
	\dom{3} &:= \{(p_1, p_2) \ : \  p_1,p_2 \ \in \Pcal \setminus \{(0,y) \}, \ y_1 \geq (1-\eps) R_n \wedge (R_n-y), 
		\ y_2 \leq R_n - y_1 \}
\end{align*}
and write, for $i = 1,2,3$,
\[
	\mathcal{I}_n^{(i)}(y) := \Exp { \sum_{(p_1,p_2)  \in \dom{i}} 
	\ind{p_1 \in \BallHyp{y}} \cdot \ind{p_2 \in \BallSym{p_1} \cap \BallHyp{y}}
	\cdot D_{\Po} (y,k_n-2;\Pcal \setminus \{ p_1, p_2\})}.
\]


In the first case, note that for $y \in \Kcal_{C}(k_n)$ we have, for small enough $\varepsilon$, $2y \le \bigO{1} 2(1+\eps)
\frac{R_n}{2\alpha +1} = \smallO{R_n}$. Thus $y_1 + y_2 \geq 2(R_n - y) = \Omega(R_n)$ and thus $p_2 \in \BallHyp{p_1}$ for large enough $n$. Furthermore, 
$y_2 > R_n - y_1 + 2\ln (\pi/2)$, which implies that $p_2 \in \BallPo{p_1}$ too. 
Hence, the contribution from these pairs is zero.   

The Campbell-Mecke formula yields that: 
\begin{equation*}
\begin{split} 
\mathcal{I}_n^{(1)}(y) 
&=O(1) \int_{-I_n}^{I_n} \int_{(1-\eps) R_n \wedge (R_n-y)}^{R_n} \ind{p_1 \in \BallHyp{y}} \times\\
& \hspace{15pt}\int_{-I_n}^{I_n} \int_{R_n - y}^{R_n} 
\ind{p_2 \in \BallSym{p_1} \cap \BallHyp{y}}
  \rho_{\Po}(y,k_n-2) \cdot
e^{-\alpha( y_2 + y_1)} \dd y_2 \dd x_2 \dd y_1 \dd x_1.
\end{split}
\end{equation*}

We proceed to bound the integral: 
\begin{align*}
	&\int_{-I_n}^{I_n} \int_{(1-\eps) R_n \wedge (R_n-y)}^{R_n} \hspace{-5pt} \ind{p_1 \in \BallHyp{y}}
		\int_{-I_n}^{I_n} \int_{R_n - y}^{R_n } \ind{p_2 \in \BallSym{p_1} \cap \BallHyp{y}} 
		e^{-\alpha(y_1 + y_2)} \dd y_2 \dd x_2 \dd y_1 \dd x_1 \\
	&\leq \int_{-I_n}^{I_n} \int_{(1-\eps) R_n \wedge (R_n-y)}^{R_n} 
		\int_{-I_n}^{I_n} \int_{R_n - y}^{R_n}  e^{-\alpha(y_1 + y_2)} \dd y_2 \dd x_2 \dd y_1 \dd x_1\\
	&= \left( \int_{-I_n}^{I_n} \int_{(1-\eps) R_n \wedge (R_n-y)}^{R_n} e^{- \alpha y_1} \dd y_1 \dd x_1 \right) 
	\left(\int_{-I_n}^{I_n} \int_{R_n - y}^{R_n}  e^{-\alpha y_2} \dd y_2 \dd x_2 \right).
\end{align*}
We evaluate
$$  \int_{-I_n}^{I_n} \int_{(1-\eps) R_n \wedge (R_n-y)}^{R_n} 
e^{- \alpha y_1} dy_1 dx_1= O(1) \cdot n \cdot e^{-\alpha R_n + ((\eps R_n) \vee y))\alpha}
=O(1) \cdot n \cdot e^{-\alpha R_n + \alpha y + \alpha \eps R_n }
$$
and 
$$\int_{-I_n}^{I_n} \int_{R_n - y}^{R_n}  e^{-\alpha y_2} dy_2 dx_2 
=O(1) \cdot n \cdot e^{-\alpha R_n +\alpha y}.
$$
Also, $n \cdot e^{-\alpha R_n} = O(1) \cdot e^{(1/2 -\alpha) R_n}$, whereby we deduce that 
\begin{equation*}
\begin{split}
&\hspace{-30pt}\int_{\dom{1}} \ind{p_1 \in \BallHyp{y}} \ind{p_2 \in \BallSym{p_1} \cap \BallHyp{y}} 
 e^{-\alpha(y_1 + y_2)} dy_2 dx_2 dy_1 dx_1 \\
&=O(1) \cdot e^{(1-2\alpha)R_n + 2\alpha y + \alpha \eps R_n} =O(1) \cdot n^{2(1-2\alpha) + 2\alpha \eps} \cdot e^{2\alpha y}.
\end{split}.
\end{equation*}

With these computations we obtain
\begin{align*} 
	\int_{\Kcal_{C}(k_n)} \mathcal{I}_n^{(1)}(y) e^{-\alpha y} \dd x \dd y
	&= O(1) n^{(2(1-2\alpha) + 2\alpha \eps}
		\int_{\Kcal_{C}(k_n)} e^{2\alpha y} \rho_{\Po}(y,k_n-2) e^{-\alpha y} \dd y \dd x \\ 
	&=O(1) n^{(2(1-2\alpha) + 2\alpha \eps} \, k_n^{2\alpha - 1}.
\end{align*}
Thus, for $1/2 < \alpha < 3/4$, we have 
\begin{equation*}
k_n^{6 \alpha -3} \,  n^{2(1-2\alpha) + 2\alpha \eps} \, k_n^{2\alpha -1} = 
n^{2\alpha \eps} \, \left( \frac{k_n^2}{n} \right)^{2(2\alpha -1)} = o(1), 
\end{equation*}
provided that $\eps = \eps (\alpha)>0$ is small enough, and hence for such $\eps$
\[
	\lim_{n \to \infty} k_n^{6\alpha - 3} \int_{\Kcal_{C}(k_n)} \mathcal{I}_n^{(1)}(y) e^{-\alpha y} \dd x \dd y = 0.
\]

When $\alpha > 3/4$ we have $2(1-2\alpha) < 1/2(4\alpha - 1)$ and we get
\begin{equation*} 
k_n^{2\alpha} \, n^{2(1-2\alpha) + 2\alpha \eps} \cdot k_n^{2\alpha -1}
\le  k_n^{4\alpha - 1} \, n^{2(1-2\alpha} n^{2\alpha \eps} = o(1),
\end{equation*}
provided that $\eps$ is small enough, depending on $\alpha$, so that
\[
	\lim_{n \to \infty} k_n^{2\alpha} \int_{\Kcal_{C}(k_n)} \mathcal{I}_n^{(1)}(y) e^{-\alpha y} \dd x \dd y = 0.
\]


We now consider the second sub-domain $\dom{2}$. The Campbell-Mecke formula yields that: 
\begin{align*}
	\mathcal{I}_n^{(2)}(y) 
	&= \Exp { \sum_{(p_1,p_2)  \in \dom{2}} \ind{p_1 \in \BallHyp{y}} \ind{p_2 \in \BallSym{p_1} \cap \BallHyp{y}}
		D_{\Po} (y,k_n-2;\Pcal \setminus \{ p_1\})} \\
	&= O(1) \rho_{\Po}(y,k_n-2) \cdot \int_{-I_n}^{I_n} \int_{(1-\eps) R_n \wedge (R_n-y)}^{R_n} 
		\ind{p_1 \in \BallHyp{y}} \times \\
	&\hspace{15pt} \int_{-I_n}^{I_n} \int_{R_n - y_1}^{R_n-y} \ind{p_2 \in \BallSym{p_1} \cap \BallHyp{y}}
		e^{-\alpha(y_1 + y_2)} \dd y_2 \dd x_2 \dd y_1 \dd x_1.
\end{align*}

We bound the integral as follows: 
\begin{align*}
	&\int_{-I_n}^{I_n} \int_{(1-\eps) R_n \wedge (R_n-y)}^{R_n} \hspace{-5pt} \ind{p_1 \in \BallHyp{y}}
		\int_{-I_n}^{I_n} \int_{R_n - y_1}^{R_n - y} \ind{p_2 \in \BallSym{p_1} \cap \BallHyp{y}}
		e^{-\alpha(y_1 + y_2)} \dd y_2 \dd x_2 \dd y_1 \dd x_1 \\
	&\leq \int_{-I_n}^{I_n} \int_{(1-\eps) R_n \wedge (R_n-y)}^{R_n} \ind{p_1 \in \BallHyp{y}}
		\int_{-I_n}^{I_n} \int_{R_n - y_1}^{R_n - y} 
		\ind{p_2 \in  \BallHyp{y}} e^{-\alpha(y_1 + y_2)} \dd y_2 \dd x_2 \dd y_1 \dd x_1.
\end{align*}
Now,
\begin{align*}
&\int_{-I_n}^{I_n} \int_{R_n - y_1}^{R_n - y} \ind{p_2 \in  \BallHyp{y}} \cdot  
e^{-\alpha y_2} dy_2 dx_2 = O(1) \cdot e^{y/2} \int_{R_n - y_1}^{R_n - y} e^{(1/2 - \alpha) y_2} dy_2 \\
&= O(1) \cdot e^{y/2 + (1/2 - \alpha) (R_n - y_1)}.
\end{align*}
We then integrate with respect to $y_1$:
\begin{align*}
	O(1) \cdot e^{y/2} \cdot 
	&\int_{-I_n}^{I_n} \int_{(1-\eps) R_n \wedge (R_n-y)}^{R_n} \ind{p_1 \in \BallHyp{y}} 
		e^{(1/2 - \alpha) (R_n - y_1)} e^{-\alpha y_1} dy_1 dx_1 \\
	&\le O(1) \cdot e^{y/2 + (1/2 -\alpha) R_n} \cdot \int_{-I_n}^{I_n} \int_{(1-\eps) R_n \wedge (R_n-y)}^{R_n} 
		e^{(\alpha -1/2) y_1} e^{-\alpha y_1} dy_1dx_1 \\
	&=O(1) \cdot e^{y/2 + (1 -\alpha) R_n -((1-\eps) R_n \wedge (R_n-y))/2} \\
	&=O(1) \cdot e^{y/2 + (1/2 -\alpha) R_n + ((\eps R_n) \vee y)/2}\\
	&=O(1) \cdot e^{y + (1/2 -\alpha) R_n + \eps R_n}
		= O(1) \cdot n^{1- 2\alpha+ \eps} \cdot e^{y}. 
\end{align*}

Therefore we get
\begin{align*} 
	&\int_{\Kcal_{C}(k_n)}\mathcal{I}_n^{(2)}(y) e^{-\alpha y} \dd x \dd y\\
	&= O\left( n^{1-2\alpha + \eps} \right)
		\int_{\Kcal_{C}(k_n)} \rho_{\Po}(y,k_n-2) e^{y} e^{-\alpha y} \dd x \dd y\\
	&= \bigO{1} n^{1-2\alpha + \eps} k_n^{-2\alpha + 1},
\end{align*}
where we used Lemma~\ref{lem:gamma_approx} with $t = 1$.

For $1/2 < \alpha < 3/4$, we have
\[
	k_n^{4 \alpha -2} \cdot  n^{1-2\alpha + \eps} = n^{\eps} \left( \frac{k_n^2}{n}\right)^{2\alpha -1} = o(1),
\]
provided that $\eps = \eps (\alpha)>0$ is small enough, yielding
\[
	\lim_{n \to \infty} k_n^{6\alpha - 3} 
			\int_{\Kcal_{C}(k_n)}\mathcal{I}_n^{(2)}(y) e^{-\alpha y} \dd x \dd y = 0.
\]
Similarly, for $\alpha > 3/4$ we have $2\alpha -1 > 1/2$ and we get
\[
	k_n \cdot n^{1-2\alpha + \eps} \ll n^{-1/2 + \eps}  \cdot k_n  = o(1),
\]
provided that $\eps$ is small enough, so that
\[
	\lim_{n \to \infty} k_n^{2\alpha}
			\int_{\Kcal_{C}(k_n)} \mathcal{I}_n^{(2)}(y) e^{-\alpha y} \dd x \dd y = 0.
\]

For the third sub-domain $\dom{3}$ we shall use~\eqref{eq:symdiff_loc} which states that if 
$p_2=(x_2,y_2) \in \BallSym{p_1}\cap \BallHyp{y}$ and $y_2\leq R_n - y_1$, then 
$y_2 \geq \hat{y}(x_1,y_1)$, where $\hat{y}(x_1,y_1) = \left(2 \log(|x_1|e^{-y_1/2}) - \log c_K \right) \vee 0$ (cf.~\eqref{eq:to_use_I}). Moreover, $p_2 \in \FatBallHyp{p_1}$.


Again, we will use the Campbell-Mecke formula: 
\begin{align*}
	\mathcal{I}_n^{(3)}(y) &= \Exp { \sum_{(p_1,p_2)  \in \dom{3}} 
		\ind{p_1 \in \BallHyp{y}} \cdot \ind{p_2 \in \BallSym{p_1} \cap \BallHyp{y}}
		\cdot D_{\Po} (y,k_n-2;\Pcal \setminus \{ p_1, p_2\})} \\
	&= O(1) \rho_{\Po}(y,k_n-2) \int_{-I_n}^{I_n} \int_{(1-\eps) R_n \wedge (R_n-y)}^{R_n} \ind{p_1 \in \BallHyp{y}}
		\times\\
	&\hspace{25pt} \int_{-I_n}^{I_n} \int_{0}^{R_n-y_1} 
		\ind{p_2 \in \BallSym{p_1} \cap \BallHyp{y}} 
		e^{-\alpha(y_1 + y_2)} dy_2 dx_2 dy_1 dx_1 \\
\end{align*}

The inner integral with respect to $p_2 := (x_2,y_2)$ is 
\begin{align*}
	&\hspace{-30pt} \int_{-I_n}^{I_n} \int_{0}^{R_n - y_1}  \ind{p_2 \in \BallSym{p_1}\cap \BallHyp{y}}  
		e^{-\alpha y_2} dy_2 dx_2\\
	&\leq \int_{-I_n}^{I_n} \int_{0}^{R_n-y_1}  \ind{y_2 \geq c(x_1,y_1), p_2 \in \FatBallHyp{(0,y)}}  
		e^{-\alpha y_2} dy_2 dx_2 \\
	&= O(1) e^{y/2} \int_{c(x_1,y_1)}^{R_n - y_1} e^{y_2/2 - \alpha y_2} dy_2 \\
	&= O(1) e^{y/2 + (1/2 -\alpha) \hat{y}(x_1,y_1)}.
\end{align*}

Thus, we get
\begin{align*}
	&\int_{-I_n}^{I_n} \int_{(1-\eps) R_n \wedge (R_n-y)}^{R_n} \ind{p_1 \in \BallHyp{y}}
		\int_{-I_n}^{I_n} \int_{0}^{R_n-y_1} 
		\ind{p_2 \in \BallSym{p_1} \cap \BallHyp{y}} \times \\ 
	& \hspace{2cm}  e^{-\alpha(y_1 + y_2)} dy_2 dx_2 dy_1 dx_1 \\
	&\leq O(1) \int_{-I_n}^{I_n} \int_{(1-\eps) R_n \wedge (R_n-y)}^{R_n} 
		e^{y/2 + (1/2 -\alpha) \hat{y}(x_1,y_1)} e^{-\alpha y_1} dy_1 dx_1. 
\end{align*}
Due to symmetry, to bound the integral it is enough to integrate this with respect to $x_1$ from 0 to $I_n$.
We will split this integral into two parts according to the value of $c(x_1,y_1)$:
\[
	\int_0^{I_n} e^{\hat{y}(x_1,y_1) (1/2 -\alpha)} dx_1 = 
	\int_{c_K e^{y_1/2}}^{I_n} e^{c(x_1,y_1) (1/2 -\alpha)} dx_1 + \int_0^{c_K e^{y_1/2}} dx_1.
\]
The first integral becomes: 
\begin{align*}
	&\hspace{-30pt}\int_{c_K e^{y_1/2}}^{I_n} e^{\hat{y}(x_1,y_1) (1/2 -\alpha)} dx_1  = 
 		O(1)\cdot \int_{c_K e^{y_1/2}}^{I_n} x_1^{1 -2\alpha} 
		e^{-\frac{y_1}{2} (1-2\alpha)} dx_1 \\
	&= \begin{cases}
		O(R_n) \cdot e^{-y_1/2 + \alpha y_1} \cdot e^{\frac{R_n}{2} 2(1-\alpha)} & \ \mbox{if $\alpha \leq 1$}\\
		O(1) \cdot e^{-y_1/2 + \alpha y_1 + 2(1-\alpha)y_1/2} & \ \mbox{if $\alpha > 1$}
		\end{cases}\\
	&=\begin{cases}
		O(R_n) \cdot e^{(\alpha -1/2) y_1} \cdot n^{2(1-\alpha)} & \ \mbox{if $\alpha \leq 1$}\\
		O(1) \cdot e^{y_1/2} &\ \mbox{if $\alpha > 1$}
	\end{cases}.  
\end{align*}
The second integral trivially gives: 
\[
	\int_0^{c_K e^{y_1/2}} dx_1 = O(1) \cdot e^{y_1/2}.
\]
Putting these two together we conclude that 
\[
	e^{y/2} \cdot \int_0^{I_n} e^{c(x_1,y_1) (1/2 -\alpha)} dx_1
	= O(1) \cdot e^{y_1/2 +y(3/2-\alpha)}.
\]

Now, we integrate these with respect to $y_1$:
\begin{align*}
	&n^{2(1-\alpha)} \cdot \int_{(1-\eps) R_n \wedge (R_n-y)}^{R_n} e^{(\alpha -1/2)y_1 - \alpha y_1} dy_1 
		=O(1) \cdot n^{2(1-\alpha)} \cdot e^{-R_n/2 + \eps R_n/2  + y/2}  \\
	&= O(1) \cdot n^{1-2\alpha + \eps} \cdot e^{y/2}.
\end{align*}
%Also,
%\begin{equation}
%\begin{split}
%&\int_{(1-\eps) R_n \wedge (R_n-y)}^{R_n} e^{(1/2 - \alpha) y_1} dy_1 
%= O(1) \cdot e^{(1/2-\alpha) R_n + y(\alpha -1/2) + \eps R_n (\alpha -1/2)} \\
%&= O(1) \cdot n^{1-2\alpha +\eps(2\alpha -1)} \cdot e^{y(\alpha -1/2)}. 
%\end{split}
%\end{equation}
Therefore, we conclude that
\[
	\mathcal{I}_n^{(3)}(y) = \bigO{R_n} n^{1-2\alpha +\eps(2\alpha -1)} \, e^{y/2} \rho_{\Po}(y,k_n-2)
\]
and hence, using again Lemma~\ref{lem:gamma_approx},
\begin{align*}
	\int_{\Kcal_{C}(k_n)} \mathcal{I}_n^{(3)}(y) e^{-\alpha y} \dd x \dd y
	&= \bigO{R_n} n^{1-2\alpha +\eps(2\alpha -1)} \int_{\Kcal_{C}(k_n)} e^{y/2} 
		\rho_{\Po}(y,k_n-2) e^{-\alpha y} \dd x \dd y\\
	&= \bigO{R_n} n^{1-2\alpha +\eps(2\alpha -1)} k_n^{-2\alpha + 1}.
\end{align*}

It follows that for $\eps = \eps(\alpha)$ small enough
\[
	k_n^{6\alpha - 3} R_n n^{1-2\alpha +\eps(2\alpha -1)} k_n^{-2\alpha + 1}
	= R_n n^{\eps(2\alpha -1)} \left(\frac{k_n^2}{n}\right)^{2\alpha - 1} = \smallO{1}
\]
and hence for $\alpha > 1/2$,
\[
	\lim_{n \to \infty} k_n^{6\alpha - 3} \int_{\Kcal_{C}(k_n)} \mathcal{I}_n^{(3)}(y) e^{-\alpha y} \dd x \dd y = 0.
\]
Since $2\alpha - 1 \ge 1/2$ when $\alpha \ge 3/4$ it immediately follows that
\[
	\lim_{n \to \infty} k_n^{2\alpha} \int_{\Kcal_{C}(k_n)} \mathcal{I}_n^{(3)}(y) e^{-\alpha y} \dd x \dd y = 0.
\]

\paragraph{The sums~\eqref{eq:sum_5} and~\eqref{eq:sum_5b}}

Again, the analysis for both terms are similar and we shall analyze~\eqref{eq:sum_5}. Let us set $p=(0,y)$. Recall that $\BallSym{y}\cap \Rcal([R_n - y + 2 \log\left(\frac{\pi}{2}\right),R_n]) = \emptyset$. Thus, the summand in~\eqref{eq:sum_5} is equal to 0, when $y_1 > R_n - y + 2 \log (\pi/2)$. 

Recall the definition of the extended ball $\FatBallHyp{p}$ around $p$~\eqref{eq:def_fatball} that contains both $\BallHyp{p}$ and $\BallPo{p}$ 
\[
	\FatBallHyp{p} : = \{ p^\prime : y^\prime < R_n - y, \ |x^\prime| < (1+K) e^{\frac{1}{2} (y + y^\prime)}  \},
\]
and that we have $\Exp{D_{\Po} (y,k_n-2;\Pcal \setminus \{ p_1,p_2\})} = \rho_{\Po}(y,k_n-2)$.

Further, observe that,
\begin{equation*} %\label{eq:ball_inclusion} 
\BallHyp{p} \cap \Rcal ([0,R-y)) \subseteq \FatBallHyp{p}
\end{equation*}
and 
\begin{equation*} %\label{eq:ball_inclusion_lower}
\BallHyp{p} \cap \Rcal ([R-y,R_n]) = \Rcal ([r(p),R_n]).
\end{equation*}
We thus conclude that 
\begin{equation} \label{eq:ball_inclusions}
\BallHyp{p} \subseteq \FatBallHyp{p} \cup \Rcal([r(p),R_n]).
\end{equation}
Hence, if we set 
\[
h_y(p_1, \Pcal) := \ind{p_1 \in \BallHyp{p}\setminus \BallPo{p}} \cdot    
\left( \mu \left( \FatBallHyp{p_1} \cap \FatBallHyp{p} \right)
+ \mu \left( \Rcal([R_n-y,R_n]) \right) \right),
\]
then 
\begin{align*}
&\ind{p_1\in \BallHyp{p}\setminus \BallPo{p}} \cdot \Exp{ \left(\sum_{p_2 \in \Pcal \setminus 
\{p,p_1\}} \ind{p_2 \in \BallHyp{p} \cap \BallPo{p_1}}\right) \cdot 
D_{\Po} (y,k_n-2;\Pcal \setminus \{ p_1, p_2\})
} \\
&=O(1)\cdot
\ind{p_1\in \BallHyp{p}\setminus \BallPo{p}} \cdot \mu (\BallHyp{p}\cap \BallHyp{p_1}) \rho_{\Po}(y,k_n-2) \\
& \leq O(1) \cdot  h_y (p_1, \Pcal) \rho_{\Po}(y,k_n-2) . 
\end{align*}
To calculate the expectation of the above function we need to approximate the 
intersection of the two balls $\FatBallHyp{p}$ and $\FatBallHyp{p_1}$, 
where $p_1= (x_1,y_1)$. 
Let us assume without loss of generality that $x_1 > 0$. 
The right boundary of $\FatBallHyp{p}$ is given by the equation 
$x = x(y_1) = (1+K)e^{\frac{1}{2} (y + y_1)}$ whereas the left boundary of $\FatBallHyp{p_1}$ is given by the curve $x = x(y_1)= x_1 - (1+ K)e^{\frac{1}{2} (y + y_1)}.$ 

The equation that determines the intersecting point of the two curves is
\[
	x_1 - (1+K)e^{(\hat{y} + y_1)/2}= (1+K) e^{(\hat{y} + y)/2},
\]
where $\hat{y}$ is the $y$-coordinate of the intersecting point. 
We can solve the above for $\hat{y}$  
\begin{equation*} 
\begin{split}
x_1 &=(1+K) e^{\hat{y}/2} \left( e^{y/2} + e^{y_1/2} \right).
\end{split}
\end{equation*}
But since $p_1=(x_1,y_1)  \in \BallSym{p}$, we also have $x_1 > e^{\frac{y + y_1}{2}}$. Therefore, 
\begin{equation*}
\begin{split}
 e^{\hat{y}/2}& > \frac{1}{1+K}~\frac{e^{\frac{y + y_1}{2}}}{ e^{y/2}+ e^{y_1/2}} \geq 
\frac{1}{2(1+K)}~\frac{e^{\frac{y_1 + y}{2}}}{ e^{(y \wedge y_1) /2}} 
> \frac{1}{2(1 + K)} ~ e^{(y \vee y_1)/2}. 
 \end{split}
\end{equation*}
The above yields
\begin{equation} \label{eq:to_use_I}
\hat{y} > (y \vee y_1) - 2\log(2(1+K)) := \hat{y}(y_1, y). 
\end{equation}
which, in turn, implies the following 
\begin{equation}\label{eq:intersex_approx}
	p \in \FatBallHyp{(0,y)}\cap \FatBallHyp{p_1} \Rightarrow y(p) \ge \hat{y}(y_1,y).
\end{equation}
We thus conclude that 
\[ 
	\BallHyp{p_1} \cap \BallHyp{p} \subseteq \left(\FatBallHyp{p} \cap \Rcal([\hat{y}(y_1,y), R_n])\right)
	\, \cup \, \Rcal ([R_n - y,R_n]),
\]
which in turn implies that
\[
	\mu \left( \FatBallHyp{p_1} \cap \BallHyp{p} \right) \leq 
	\mu\left( \FatBallHyp{p} \cap  \Rcal([\hat{y}(y_1,y), R_n]\right) + 
	\mu (\Rcal ([R_n - y, R_n]) ).
\]
Therefore, 
\begin{align*} 
	h_y(p_1, \Pcal) &\leq \ind{p_1 \in \BallHyp{p}\setminus \BallPo{p}} 
    	\mu  \left( \FatBallHyp{p} \cap  \Rcal([\hat{y}(y_1,y), R_n])\right)
        \\
	&\hspace{10pt}+ \ind{p_1 \in \BallHyp{p}\setminus \BallPo{p}}
    	\mu  \left( \Rcal ([R_n - y, R_n]) \right).
\end{align*}



Now, the Campbell-Mecke formula gives
\begin{align*}
	&\Exp{ \sum_{p_1,p_2 \in\Pcal \setminus \{(0,y)\} \atop y(p_1) \geq K} 
		\hspace{-10pt} \ind{p_1\in \BallHyp{y}\setminus \BallPo{y}} \ind{p_2\in \BallHyp{y}\cap \BallPo{y}} 
		\, D_{\Po} (y,k_n-2;\Pcal \setminus \{ p_1,p_2\})}\\
	&\le \Exp{\left( \sum_{p_1 \in \Pcal} 
		h_y (p_1, \Pcal \setminus \{ p_1 \})\right)} \\
	&=\frac{\nu \alpha}{\pi} \int_{\Rcal} \Exp{h_y(p_1, \Pcal \setminus \{p_1\})}
		e^{-\alpha y_1} \dd x_1 \dd y_1\\
	&\le \frac{\nu \alpha}{\pi} \int_{\Rcal} \ind{p_1 \in \BallHyp{p}\setminus \BallPo{p}} 
	    	\mu  \left( \FatBallHyp{p} \cap  \Rcal([\hat{y}(y_1,y), R_n])\right)
	    	e^{-\alpha y_1} \dd x_1 \dd y_1 \numberthis \label{eq:h_upper_bound_1}\\
	&\hspace{10pt}+ \frac{\nu \alpha}{\pi} \int_{\Rcal} \ind{p_1 \in \BallHyp{p}\setminus \BallPo{p}}
	    	\mu  \left( \Rcal ([R_n - y, R_n]) \right) e^{-\alpha y_1} \dd x_1 \dd y_1.
	    	\numberthis \label{eq:h_upper_bound_2}
\end{align*}

Recall that $(\BallSym{(0,y)})\cap \Rcal([R_n - y + 2 \log\left(\frac{\pi}{2}\right),R_n]) = \emptyset$. 
We will first calculate the measures $\mu$ appearing in~\eqref{eq:h_upper_bound_1} and~\eqref{eq:h_upper_bound_2}. The first one is:
\begin{align*}
	\mu\left( \FatBallHyp{y} \cap  \Rcal([c(y_1,y), R_n])\right) 
	&\leq (1+ K) \frac{\nu \alpha}{\pi} \cdot e^{y/2}  \int_{c(y_1,y)}^{R_n} e^{-(\alpha - \frac{1}{2}) y'} \, dy' \\
	&=  \bigO{e^{\frac{y}{2} - (\alpha-\frac{1}{2}) (y \vee y_1)}}.
\end{align*}

The second term is: 
\begin{align*}
	\mu \left( \Rcal([R_n - y,R_n]) \right) 
    &= \frac{\nu \alpha}{\pi} \int_{R_n - y}^{R_n} \pi e^{\frac{R_n}{2}} e^{-\alpha y'} \, dy' 
    	= \bigO{e^{\frac{R_n}{2}} e^{-\alpha (R_n-y)}} = \bigO{e^{\alpha y - (\alpha - \frac{1}{2})R_n}}. 
\end{align*}

Using these, we get
\begin{align} 
	&\int_{\Rcal ([0, R_n - y_n + 2 \ln \frac{\pi}{2}])} \Exp{h_y(p_1, \Pcal \setminus \{p_1\})} 
    e^{-\alpha y_1} \, dx_1 \, dy_1 \notag \\
	&= \bigO{1} \int_{\Rcal ([0, R_n - y+ 2 \ln \frac{\pi}{2}])} \ind{p_1 \in \BallSym{p}} 
		e^{\frac{y}{2} - (\alpha - \frac{1}{2}) (y \vee y_1) - \alpha y_1} \, dx_1 \, dy_1
		\label{eq:Mecke_sum_1}\\ 
	&\hspace{10pt}+ \bigO{1} \int_{\Rcal ([0, R_n - y + 2 \ln \frac{\pi}{2}])} 
    	\ind{p_1 \in \BallHyp{(0,y)}} 
    	e^{\alpha y - (\alpha - \frac{1}{2})R_n - \alpha y_1} \, dx_1 \, dy_1.\label{eq:Mecke_sum_2}
\end{align}
Now, Lemma~\ref{lem:asymptotics_Omega_hyperbolic} implies that 
for any $y \in [0, R_n - y_n + 2 \ln \frac{\pi}{2}]$, we have 
\[ 
	\int_{-I_n}^{I_n} \ind{p_1 \in \BallSym{y}} \, dx_1 \leq 2 K e^{\frac{3}{2} (y_1 + y) - R_n}.
\]

Therefore, \eqref{eq:Mecke_sum_1} is 
\begin{align*}
	&\hspace{-20pt}O(1) \cdot e^{2 y - R_n} \int_{0}^{R_n - y + 2 \ln \frac{\pi}{2}} 
    	e^{\frac{3y_1}{2} - (\alpha - \frac{1}{2})(y_1 \vee y) - \alpha y_1} \, dy_1 \\
 	&=  O(1) \cdot e^{2 y - R_n} \left( \int_{0}^{y} e^{\frac{3y_1}{2} - (2\alpha - \frac{1}{2})y_1} \, dy_1 
 		+ e^{-(\alpha-\frac{1}{2}) y}\int_{y}^{R_n - y + 2 \ln \frac{\pi}{2}} e^{(\frac{3}{2} - \alpha) y_1} \, dy_1 \right)\\
  	&= O(1) \left( 
	\begin{cases}
	e^{(4-2\alpha) y - R_n}, & \mbox{if $\alpha < 2$} \\
	R_n \cdot e^{2y - R_n}, & \mbox{if $\alpha \geq 2$}
	\end{cases}
	+
	\begin{cases}
	e^{-(\alpha - \frac{1}{2})R_n +y}, & \mbox{if $\alpha < 3/2$} \\
	R_n \cdot  e^{2(2-\alpha)y - R_n}, & \mbox{if $\alpha \geq 3/2$}
	\end{cases}\right).
\end{align*}

Similarly, for~\eqref{eq:Mecke_sum_2} we have
\begin{align*}
	&\hspace{-30pt}\int_{\Rcal ([0, R_n - y + 2 \ln \frac{\pi}{2}])} \ind{p_1 \in \BallSym{(0,y)}} e^{\alpha y - (\alpha - \frac{1}{2})R_n - \alpha y_1} \, dx_1 \, dy_1\\
	&= e^{\frac{3y}{2} - R_n + \alpha y - (\alpha - \frac{1}{2})R_n} 
    	\cdot \int_{0}^{R_n - y + 2 \ln \frac{\pi}{2}} e^{\frac{3y_1}{2}-\alpha y_1} \, dy_1\\
	&= O(1)\cdot 
	\begin{cases} 
	e^{\frac{3y}{2} - R_n + \alpha y - (\alpha - \frac{1}{2})R_n + (\frac{3}{2} - \alpha)(R_n-y)} 
	, & \mbox{if $\alpha < 3/2$} \\ 
	R_n \cdot e^{(\frac{3}{2} +\alpha)y -  (\alpha + \frac{1}{2})R_n}, & \mbox{if 
	$\alpha \geq 3/2$}
	\end{cases}	\\
	&= O(1) \cdot 
	\begin{cases}
	  e^{-(2\alpha-1) R_n + 2 \alpha y}, & \mbox{if $\alpha < 3/2$} \\
	  R_n \cdot e^{(\frac{3}{2} +\alpha)y -  (\alpha + \frac{1}{2})R_n}, & \mbox{if 
	$\alpha \geq 3/2$}
	\end{cases}.
\end{align*}

We thus conclude, using $2(2 - \alpha)y \le y$ for $\alpha > 3/2$, that 
\begin{equation} \label{eq:upper_bound_faulty_edges} 
\Exp{\left( \sum_{p_1 \in \Pcal \setminus\{p\}} 
		h_y (p_1, \Pcal \setminus \{ p_1 \})\right)} \le \bigO{1} \cdot 
\left( \mathcal{I}_n^{(1)}(y) + \mathcal{I}_n^{(2)}(y) + \mathcal{I}_n^{(3)}(y) \right),
\end{equation}
where 
\begin{align*}
 \mathcal{I}_n^{(1)}(y) &= \begin{cases}
	e^{(4-2\alpha) y - R_n}, & \mbox{if $\alpha < 2$} \\
	R_n \cdot e^{2y - R_n}, & \mbox{if $\alpha \geq 2$}
	\end{cases},  \\
	\mathcal{I}_n^{(2)}(y) &= 
	\begin{cases}
	e^{-(\alpha - \frac{1}{2})R_n +y}, & \mbox{if $\alpha < 3/2$} \\
	R_n \cdot  e^{y - R_n}, & \mbox{if $\alpha \geq 3/2$}
	\end{cases}\\
	\mathcal{I}_n^{(3)}(y) &= 
\begin{cases}
	  e^{-(2\alpha-1) R_n + 2 \alpha y}, & \mbox{if $\alpha < 3/2$} \\
	  R_n \cdot e^{(\frac{3}{2} +\alpha)y -  (\alpha + \frac{1}{2})R_n}, & \mbox{if 
	$\alpha \geq 3/2$}
	\end{cases}.
\end{align*}

We proceed to calculate:
\begin{align*}
&\int_{\Kcal_{C}(k_n)} \Exp{\left( \sum_{p_1 \in \Pcal} h_y (p_1, \Pcal \setminus \{ p_1 \})\right)} \cdot 
 \rho_{\Po}(y,k_n-2) e^{-\alpha y} \dd y \dd x.
\end{align*}
%Firstly, note that as $\expH = \Theta (1) \cdot n \cdot k_n^{-(2\alpha +1)}$, we have 
%$$ {k_n\choose 2}^{-1}\cdot \expH^{-1} = O(1) \cdot \frac{k_n^{2\alpha-1}}{n}.$$
%Also, $\Exp{\left( \sum_{p_1 \in \Pcal} h_y (p_1, \Pcal \setminus \{ p_1 \})\right)}$ is given 
%as the sum of $\mathcal{I}_n^{(1)}(y), \mathcal{I}_n^{(2}(y)$ and $\mathcal{I}_n^{(3)}(y)$ (cf.~\eqref{eq:upper_bound_faulty_edges}). Setting 
%\[
%	J_3 = \frac{k_n^{2\alpha-1}}{n}\cdot  \left(M_1+ M_2 + M_3 \right)
%\]
%with
For this we define
\[
	M_i = \int_{\Kcal_{C}(k_n)} \mathcal{I}_n^{(i)}(y) \rho_{\Po}(y,k_n-1) e^{-\alpha y} \dd y \dd x
\]
so that
\begin{align*}
\int_{\Kcal_{C}(k_n)} \Exp{\left( \sum_{p_1 \in \Pcal \setminus \{(0,y)\}} h_y (p_1, \Pcal \setminus \{ p_1 \})\right)} \rho_{\Po}(y,k_n-1) e^{-\alpha y} \dd y \dd x 
&= \bigO{M_1 + M_2 + M_3}.
\end{align*}

Computing each of the integral separately we obtain, using Lemma~\ref{lem:gamma_approx} and the fact that $n = \nu e^{R_n/2}$,
\begin{align*} 
M_1:= \int_{\Kcal_{C}(k_n)} \mathcal{I}_n^{(1)}(y) \rho_{\Po}(y,k_n-1) e^{-\alpha y} \dd y \dd x
&= O(1) \cdot 
\begin{cases} 
	\frac{k_n^{7-6\alpha}}{n^2}, & \mbox{if $\alpha <2$} \\
	R_n \frac{k_n^{3-2\alpha}}{n^2}, & \mbox{if $\alpha \geq 2$}
\end{cases}. 
\end{align*}
\begin{align*} 
M_2:= \int_{\Kcal_{C}(k_n)} \mathcal{I}_n^{(2)}(y) \rho_{\Po}(y,k_n-1) e^{-\alpha y} dy
&= O(1) \cdot 
\begin{cases}
\frac{k_n^{1-2\alpha}}{n^{2\alpha-1}}, & \mbox{if $\alpha < 3/2$} \\
R_n   \frac{k_n^{1-2\alpha}}{n^{2}}, & \mbox{if $\alpha \geq 3/2$}
\end{cases}
\end{align*}
and finally 
\begin{align*} 
M_3:= \int_{\Kcal_{C}(k_n)} \mathcal{I}_n^{(3)}(y) \rho_{\Po}(y,k_n-1) e^{-\alpha y} dy
&= O(1) \cdot 
\begin{cases} 
\frac{k_n^{2\alpha - 1}}{n^{4\alpha -2}}, & \mbox{if $\alpha <3/2$} \\ 
R_n \cdot \frac{k_n^2}{n^{2\alpha+1}}, &\mbox{if $\alpha \geq 3/2$}
\end{cases}  .
\end{align*}

Now, we will consider the two cases according to the value of $\alpha$. First we note that $R_n = \bigO{\log(n)}$ and since
$k_n = O(n^{\frac{1}{2\alpha +1}})$ and $\alpha > 1/2$ we have that $R_n k_n^2 n^{-1} = \smallO{1}$.
Assume first that $1/2 < \alpha < 3/4$. In this case, we want to show that 
\begin{equation} \label{eq:int3_to_prove_I}
\lim_{n \to \infty} k_n^{6\alpha -3} (M_1 + M_2 + M_3) = 0. 
\end{equation}
Using the above expression for $M_i$, we have 
\begin{align*} 
 k_n^{6\alpha -3} (M_1 + M_2 + M_3) &= O(1) \cdot  
 k_n^{6\alpha -3} 
\left( 
\frac{k_n^{7-6\alpha}}{n^2} + \frac{k_n^{1-2\alpha}}{n^{2\alpha-1}} 
+\frac{k_n^{2\alpha-1}}{n^{4\alpha - 3}}.
\right) 
\end{align*}
We wish to show that each one of the above three terms is $o(1)$ for $k_n = O(n^{\frac{1}{2\alpha +1}})$. 
For the first one we have 
\[ 
	k_n^{6\alpha -3} \frac{k_n^{7-6\alpha}}{n^2} = \left(\frac{k_n^{2}}{n}\right)^2 = \smallO{1}. 
\]
The second term yields: 
\[
	k_n^{6\alpha -3}  \frac{k_n^{-2\alpha+1}}{n^{2\alpha-1}} = \left(\frac{k_n^2}{n}\right)^{2\alpha - 1} = \smallO{1}.
\]
Finally, the third one yields: 
\[
	k_n^{6\alpha -3} \cdot \frac{k_n^{2\alpha -1}}{n^{4\alpha - 2}}  
	= \left(\frac{k_n^2}{n}\right)^{4\alpha - 2} = \smallO{1}.
\]
 
For $\alpha \ge 3/4$, we would like to show that 
\begin{equation} \label{eq:int3_to_prove_II}
\lim_{n \to \infty} k_n^{2\alpha} \cdot (M_1 + M_2 + M_3) = 0. 
\end{equation}
Firstly, we note that $M_i$ as above if $3/4 < \alpha < 3/2$. Therefore, since for this range $2 \alpha <6\alpha - 3$ the result follows from the above analysis. Next, if we consider $3/2 \leq \alpha <2$, it is only $M_2$ and $M_3$ that change values. In particular, for any $\alpha \geq 3/2$ we have 
\[
	\frac{k_n}{n}  M_2 =O(1) R_n \frac{k_n}{n^2} = o(1).
\]
Also,
\[
	k_n^{2\alpha }  M_3 = O(1) R_n \frac{k_n^{2\alpha + 2}}{n^{2\alpha + 1}}
	= R_n \smallO{\frac{n^{\alpha + 1}}{n^{2\alpha +1}}} = o(1),
\]
since $k_n = o(n^{1/2})$. Now if $\alpha \geq 2$, then $M_1$ changes value and we have 
\[
	k_n^{2\alpha} M_1 =O(1) R_n \frac{k_n^3}{n^2} = o(1),
\] 
which finished the proof for~\eqref{eq:sum_5}.



\paragraph{The sum of~\eqref{eq:sum_6}}
%Now, we will first give an upper  bound on the term
%\begin{equation*}
%\Exp { \sum_{p_1,p_2 \in\Pcal \setminus \{(0,y)\} \atop y_1 < K}
%\ind{p_1\in \BallSym{y}} \ind{p_2\in \BallHyp{y}\cap \BallPo{y}}}.
%\end{equation*}
Using the Campbell-Mecke formula, we write 
\begin{align*}
	&\hspace{-30pt}\Exp { \sum_{p_1,p_2 \in\Pcal \setminus \{(0,y)\}, \ y_1 < K}
		\ind{p_1\in \BallSym{y}} \ind{p_2\in \BallHyp{y}\cap \BallPo{y}}}\\
	&\leq \int_{-I_n}^{I_n} \int_0^K \int_{-I_n}^{I_n} \int_0^{R_n} 
		\ind{p_1 \in \BallSym{y}} 
 		\ind{p_2 \in \BallHyp{y}\cap \BallPo{y}} e^{-\alpha y_2} e^{-\alpha y_1} \dd x_2 \dd y_2 \dd x_1 \dd y_1 \\
 	&\leq  \mu (\BallHyp{y})\cdot \int_{-I_n}^{I_n} \int_0^K \ind{p_1 \in \BallSym{y}} 
		e^{-\alpha y_1}  dx_1 dy_1.
\end{align*}
Recall that by Lemma~\ref{lem:average_degree_hyperbolic}, $\Mu{\BallHyp{y}} =O(1) e^{y/2}$. We bound the integral using Lemma~\ref{lem:asymptotics_Omega_hyperbolic}. In particular,~\eqref{eq:asymp1} implies that if $(x_1,y_1) \in \BallSym{(0,y)}$, then because $y_1 < K$
\[
	|x_1 - e^{(y+y_1)/2} |\leq e^{(y+y_1)/2} \cdot K e^{y+y_1- R_n} = O(1) 
	e^{(y+y_1)/2} \cdot e^{y- R_n}.
\]
Therefore, 
\[
	\int_{-I_n}^{I_n} \int_0^K \ind{(x_1,y_1) \in \BallSym{(0,y)}} 
	e^{-\alpha y_1}  dx_1 dy_1 = O(1) \cdot e^{y-R_n} 
	\cdot \int_0^K e^{(y+y_1)/2} 
	e^{-\alpha y_1}   dy_1 = O(1)\cdot e^{3y/2 - R_n}, 
\]
and hence
\begin{align*}
	\Exp { \sum_{p_1,p_2 \in\Pcal \setminus \{(0,y)\} \atop y_1 < K}
	\ind{p_1\in \BallSym{y}} \ind{p_2\in \BallHyp{y}\cap \BallPo{y}}} 
	= O(1) \cdot e^{2y - R_n}.
\end{align*}
Now, we integrate this over $y$ to obtain that
\begin{align*}
	&\hspace{-30pt}\int_{\Kcal_{C}(k_n)} \Exp { \sum_{p_1,p_2 \in\Pcal \setminus \{(0,y)\} \atop y_1 < K}
		\ind{p_1\in \BallSym{y}} \ind{p_2\in \BallHyp{y}\cap \BallPo{y}}} e^{-\alpha y} \dd y\\
	&= O(1) e^{-R_n} \int_{\Kcal_{C}(k_n)} e^{2y -\alpha y} \dd y
		= O(1) n^{-2}
			\begin{cases}
			k_n^{4-2\alpha)}, & \mbox{if $\alpha < 2$} \\
			\log k_n, & \mbox{if $\alpha =2$} \\
			1, & \mbox{if $\alpha >2$}
			\end{cases}.
\end{align*}

To finish the argument assume first that $1/2 <\alpha \leq 3/4$. In this case,
\[
	k_n^{6\alpha -3}  n^{-2} k_n^{4 - 2\alpha} = 
	n^{-2} \cdot k_n^{4\alpha +1} = \smallO{1}.
\]
For $3/4 \le \alpha < 2$ we use that $2\alpha < 6\alpha - 3$, so that $k_n^{2\alpha} n^{-2} k_n^{4 - 2\alpha} = \smallO{1}$. Finally, when $\alpha \geq 2$, we have that
\[
	k_n^{2\alpha} (\log(k_n) \wedge 1) n^{-2} \le k_n^{2\alpha + 1} n^{-2} = \bigO{n^{-1}} = \smallO{1}.
\] 
which completes the proof for~\eqref{eq:sum_6} and thus the proof of Proposition~\ref{prop:couling_c_H_P}.
\end{proof}

\subsection{Coupling $G_n$ to $\GPo$}\label{ssec:coupling_H_HP}

Now that we have established the equivalence of the clustering function between the Poissonized KPKVB graph $\GPo$ and the finite box graph $\Gbox$ the final step is to relate the clustering function in $\GPo$ to the KPKVB graph $G_n$. As mentioned in Section~\ref{ssec:KPKVB_to_GPo_infinite_k}, this is done by moving from $c(k_n; G_n)$ to the adjusted clustering function $c^\ast(k_n; G_n)$ (Lemma~\ref{lem:clustering_ast_H}) and then to $c^\ast(k_n; \GPo)$ (Proposition~\ref{prop:clustering_ast_H_Pois}). We start with as technical lemma on the difference between vertices with degree $k_n$ in both models. Then we give the proof of Proposition~\ref{prop:clustering_ast_H_Pois} and after that we prove  Lemma~\ref{lem:clustering_ast_H}.

Let $N_n(k_n)$ and $N_\Po(k_n)$ denote, respectively, the number of vertices with degree $k_n$ in $G_n$ and $G_\Po$. We will first show that $\left|N_{n}(k_n)-N_{\Po}(k_n)\right|$ is small. Recall that we consider the standard coupling between the binomial and Poisson process to couple $G_n$ and $G_\Po$. That is, we take a sequence of i.i.d. random elements $u_1, u_2, \dots$ uniformly on the hyperbolic disk of radius $R_n$, i.e. according to the distribution \eqref{eq:def_hyperbolic_point_distribution}. Then the KPKVB graph consists of the first $n$ points and the poissonized version of the first $N \stackrel{d}{=} \Po(n)$ points ($N$ is a Poisson random variable with expectation $n$, independent of $u_1, u_2, \dots$). Under this coupling $N_n(k) = \sum_{j=1}^n \ind{\mathrm{deg}_n(u_j)=k}$ denotes the (random) number of degree $k$ vertices in the KPKVB graph $G_n$ model with $n$ vertices and $N_{\Po}(k)=\sum_{j=1}^{N} \ind{\mathrm{deg}_{\Po}(u_j)=k}$ denotes the (random) number of degree $k$ vertices in the poissonized KPKVB graph $\GPo$.

\begin{lemma}\label{lem:diff_Nk_hyperbolic_binomial_poisson}
Let $(k_n)_{n \ge 1}$ be sequence of natural numbers with $0 \leq k_n \leq n-1$ and $k_n = o(n^{\frac{1}{2\alpha+1}})$. Then
\[
	\Exp{\left|N_{n}(k_n)-N_{\Po}(k_n)\right|} = \smallO{\Exp{N_{\Po}(k_n)}} = \smallO{n k^{-(2\alpha+1)}},
\]
and in particular,
\[
	\Exp{N_{n}(k_n)} = \bigT{n k_n^{-(2\alpha + 1)}}.
\]
\end{lemma}

%\TM{ I thought it might be a good idea to show somewhere in the paper that the result of Gugelmann et al.~that
%$N_k = (1+o(1)) p_k n$ a.a.s., extends for all 
%$k \ll n^{1/(2\alpha+1)}$ -- where $p_k$ the expression in terms on $\Gamma^+, \xi$ etc. 
%Gugelmann et al.~had show it only until some small power of $n$.
%Adding this just means we have to 
%compute the leading constants for the expectation $\Ee N_k$, and bound the variance, in addition to what we do here.
%Somehow I had expected we would include that. In fact Markus had prepared a short write up of the variance part of the argument. 
%Should not be too hard to add the rest, no? \\
%We already have the degree sequence for the infinite model somewhere in the paper.
%}
%\PvdH{Will happen but I honestly had no time to put it in.}

\begin{proof}
The second claim follows immediately from the first and the fact that $\Exp{N_\Po(k_n)} = \bigT{n k_n^{-(2\alpha + 1)}}$, see Lemma~\ref{lem:N_k_HP_P}.

To prove the first statement let $u = (r,\theta) \in \H$. Slightly abusing notation we write $B_\H(u)$ for those $u^\prime$ such that $d_\H(u,u^\prime) \le R$ and $\mu_{\H}(y)$ for the measure of $B_\H(u)$ with respect to the $(\alpha, R)$-quasi uniform measure $g(r,\theta)$ defined in~\eqref{eq:def_quasi_uniform_density}. Note that if $u^\prime$ is sampled according to $g$, then $\mu_{\H}(u)/n$ denotes the probability that $u^\prime \in B_\H(u)$. In particular, if $G_n = G(n; \alpha, \nu)$ is a KPKVB graph and $u \in \mathcal{V}_n$, then the degree distribution of $u$, conditioned on its coordinates, is Binomial on $n-1$ trails with success probability $\mu_{\H}(u)/n$. Let us define for any $m \ge 1$, $\deg_m(u) = \mathrm{Bin}(n-1, \mu_{\H}(u)/n)$ then if $U$ is sampled according to the  $(\alpha, R)$-quasi uniform distribution~\eqref{eq:def_hyperbolic_point_distribution}, $\Prob{\deg_n(U) = k_n}$ denotes the degree distribution in $G_n$ and $\Exp{N(k_n)} = n \Prob{\deg_n(U) = k_n}$. 

Recall that we couple the KPKVB graph $G_n$ with the Poisson version $\GPo$ by considering an infinite supply of i.i.d. points $u_1, u_2, \dots$ sampled on $\H$ according to the $(\alpha, R)$-quasi uniform distribution~\eqref{eq:def_hyperbolic_point_distribution}, take $N \stackrel{d}{=} \Po(n)$, and set $\mathcal{V}_n =\{u_1, \dots, u_n\}$ and $\VPo = \{u_1, \dots, u_N\}$. Observe that $\Exp{N_\Po(k_n)} = \Exp{N \Prob{\deg_{N-1}(U) = k_n}}$, where $U$ is sampled from $(\alpha, R)$-quasi uniform distribution~\eqref{eq:def_hyperbolic_point_distribution}. It is important to note that under this coupling the only difference between $G_n$ and $\GPo$ is in the number of vertices.

Denote by $\mathcal{V}_n(k_n)$ and $\VPo(k_n)$ then set of points $u_i$ with degree $k_n$ in, respectively, $G_n$ and $\GPo$. Then the result follows if we can show that
\[
	\CExp{\sum_{i = 1}^{n \wedge N} \ind{u_i \in \mathcal{V}_n(k_n) \Delta \VPo(k_n)}}{N} 
	= \smallO{n}\CProb{\deg_{N-1}(U) = k_n}{N}
\]
We first use the Chernoff concentration result for Poisson random variable~\eqref{eq:def_chernoff_bound_poisson}, which implies that for any $C>0$ we have $N \in [n-C\sqrt{n\log n},n+C\sqrt{n \log n}]$ with probability $n^{-C^2/2}$. Since we can select $C > 0$ arbitrarily large it follows that we only need to prove the result condition on the event $|N-n|\le C\sqrt{n \log(n)}$. Note that if $N = n$ then 
\[
	\sum_{i = 1}^{n \wedge N} \ind{u_i \in \mathcal{V}_n(k_n) \Delta \VPo(k_n)} = 0.
\]
Hence if we denote by $A_n^-$ the event that $n - C\sqrt{n \log(n)} \le N < n$ and by $A_n^+$ the event that $n < N \le n + C\sqrt{n \log(n)}$, we need to show that
\[
	\CExp{\sum_{i = 1}^{n \wedge N} \ind{u_i \in \mathcal{V}_n(k_n) \Delta \VPo(k_n)}}{A_n^\pm} 
		= \smallO{n}\CProb{\deg_N(U) = k_n}{A_n^\pm}
\]

We first consider the case where $n < N \le n + C\sqrt{n \log(n)}$, i.e. the case where we have more vertices in $\GPo$ than in $G_n$. The prove for the other case is similar and we omit it. Let $W = \{u_{n+1}, \dots, u_N\}$ be the set of vertices in $\GPo$ that are not in $G_n$. To ease notation we let $\mathbb{E}_n$ and $\mathbb{P}_n$ denote the conditional expectation and probability on the event $A_n^+$. First we note that
\begin{align*}
	\Expn{\sum_{i = n+1}^N \ind{u_i \in \mathcal{V}_n(k_n) \Delta \VPo(k_n)}}
	&= \Expn{\sum_{i = n+1}^N \ind{\deg_{Po}(u_i) = k_n}} \\
	&= \frac{(N-n)}{n}\ind{A_n^+} n \Probn{\deg_N(U) = k_n} \\
	&\le C \sqrt{\frac{\log(n)}{n}} n \Probn{\deg_N(U) = k_n} \\
	&= \smallO{n} n \Probn{\deg_N(U) = k_n}. 
\end{align*}
Hence we are left to consider the sum over all vertices in $\mathcal{V}_n$.

Let $u_i \in \mathcal{V}_n(k_n)$. Then since $N > n$ we have that $u_i \in \VPo$ and hence $u_i \in \mathcal{V}_n(k_n) \Delta \VPo(k_n)$ if and only if $|B_\H(u_i) \cap W| \ge 1$ and either $u_i \in \mathcal{V}_n(k_n)$ or $u_i \in \VPo(k_n)$.
From this it follows that
\[
	\sum_{i = 1}^n \ind{u_i \in \mathcal{V}_n(k_n) \Delta \VPo(k_n)}
	= \sum_{i = 1}^n \left(\ind{\deg_n(u_i) = k_n} + \ind{\deg_\Po(u_i) = k_n} \right)\ind{|B_\H(u_i) \cap W| \ge 1}
\]
We will bound this sum further. For any $u \in \H$ let $Q_n(u)$ denote the event that $|\mu_\H(u) - k_n| \le C \sqrt{k_n \log(k_n)}$ and denote by $Q_n(u)^c$ its complement. Then, 
\begin{align*}
	&\sum_{i = 1}^n \left(\ind{\deg_n(u_i) = k_n} + \ind{\deg_\Po(u_i) = k_n} \right)\ind{|B_\H(u_i) \cap W| \ge 1}\\
	&\le \sum_{i = 1}^n \left(\ind{\deg_n(u_i) = k_n} + \ind{\deg_\Po(u_i) = k_n} \right)
		\ind{|B_\H(u_i) \cap W| \ge 1}\ind{Q_n(u_i)}\\
	&\hspace{10pt}+ \sum_{i = 1}^n \left(\ind{\deg_n(u_i) = k_n} + \ind{\deg_\Po(u_i) = k_n} \right)\ind{Q_n(u_i)^c}.
\end{align*}
The second sum splits into two parts. The expectation of the the first part yields
\begin{align*}
	\Expn{\sum_{i = 1}^n \ind{\deg_n(u_i) = k_n} \ind{Q_n(u_i)^c}}
	&= n \, \Prob{\mathrm{Bin}\left(n-1, \frac{\mu_\H(U)}{n}\right) = k_n, Q_n(U)^c}\\
	&= \bigO{n k_n^{-\frac{C^2}{3}}} = \smallO{n k_n^{-(2\alpha + 1)}},
\end{align*}
where we used Lemma~\ref{lem:concentration_height_binomial}, for any $C^2/3 > 2\alpha + 1$.
Similarly, Lemma~\ref{lem:concentration_height_binomial} implies that the expectation of the second part is
\begin{align*}
	\Expn{\sum_{i = 1}^n \ind{\deg_\Po(u_i) = k_n} \ind{Q_n(u_i)^c}}
	&= n \, \Prob{\mathrm{Bin}\left(N-1, \frac{\mu_\H(U)}{n}\right) = k_n, Q_n(U)^c}\\
	&= \bigO{n k_n^{-\frac{C^2}{3}}} = \smallO{n k_n^{-(2\alpha + 1)}}.
\end{align*}

Hence it remains to show that 
\[
	\Expn{\sum_{i = 1}^n \left(\ind{\deg_n(u_i) = k_n} + \ind{\deg_\Po(u_i) = k_n} \right)
			\ind{|B_\H(u_i) \cap W| \ge 1}\ind{Q_n(u_i)}} = \smallO{n k_n^{-(2\alpha + 1)}}.
\] 
We will show that
\[
	Expn{\sum_{i = 1}^n \ind{\deg_n(u_i) = k_n}
				\ind{|B_\H(u_i) \cap W| \ge 1}\ind{Q_n(u_i)}} = \smallO{n k_n^{-(2\alpha + 1)}}.
\]
The other term follows using almost identical arguments and is omitted. To prove the above result we note that since $N > n$ we have $\ind{|B_\H(u_i) \cap W| \ge 1} \le \sum_{j = n + 1}^N \ind{u_j \in B_\H(u_i)}$. Thus
\begin{align*}
	&\hspace{-20pt}\Expn{\sum_{i = 1}^n \ind{\deg_n(u_i) = k_n}\ind{|B_\H(u_i) \cap W| \ge 1}\ind{Q_n(u_i)}}\\
	&\le\sum_{i = 1}^n \sum_{j = n+1}^N \Expn{\ind{\deg_n(u_i) = k_n}\ind{Q_n(u_i)}\ind{u_j \in B_\H(u_i)}}\\
	&= n(N-n) \Probn{\mathrm{Bin}\left(n-2, \frac{\mu_\H(U_1)}{n}\right) = k_n, Q_n(U_1), U_2 \in B_\H(U_1)}\\
	&= n(N-n) 
		\Expn{\ind{Q_n(U_1)}
			\CProb{\mathrm{Bin}\left(n-2, \frac{\mu_\H(U_1)}{n}\right) = k_n, U_2 \in B_\H(U_1)}{A_n^+, U_1}}\\
	&= n(N-n)
		\Expn{\ind{Q_n(U_1)}\CProb{\mathrm{Bin}\left(n-2, \frac{\mu_\H(U_1)}{n}\right) = k_n}{A_m^+, U_1} \frac{\mu_{\H}(U_1)}{n}},
\end{align*}
where the last step follows since, conditional on $U_1$, the events $\mathrm{Bin}\left(n-2, \frac{\mu_\H(U_1)}{n}\right) = k_n$ and $U_2\in B_\H(U_1)$ are independent and $\CProb{U_2\in B_\H(U_1)}{U_1} = \mu_\H(U_1)/n$. In addition, on the event $Q_n(U_1)$ we have that $\mu_\H(U_1) \le k_n + C \sqrt{k_n \log(k_n)}$ and thus the second statement of Lemma~\ref{lem:concentration_height_binomial} implies that
\begin{align*}
	&\Expn{\ind{Q_n(U_1)}\CProb{\mathrm{Bin}\left(n-2, \frac{\mu_\H(U_1)}{n}\right) = k_n}{A_n^+, U_1}}\\
	&= (1+\smallO{1})\Expn{\ind{Q_n(U_1)}\CProb{\mathrm{Bin}\left(N-1, \frac{\mu_\H(U_1)}{n}\right) = k_n}{A_n^+, U_1}}\\
	&\le \bigO{1}\Probn{\mathrm{Bin}\left(N-1, \frac{\mu_\H(U_1)}{n}\right) = k_n}\\
	&= \bigO{1}\Probn{\deg_N(U_1) = k_n}.
\end{align*}
Hence, we obtain
\begin{align*}
	&\hspace{-20pt}\Expn{\sum_{i = 1}^n \ind{\deg_n(u_i) = k_n}\ind{|B_\H(u_i) \cap W| \ge 1}\ind{Q_n(u_i)}}\\
	&\le (N-n)\left(k_n + C \sqrt{k_n \log(k_n)}\right) \Probn{\deg_N(U_1) = k_n}\\
	&= (1+\smallO{1}) \sqrt{n \log(n)} k_n \Probn{\deg_N(U_1) = k_n}\\
	&= \smallO{n \Probn{\deg_N(U_1) = k_n}} = \smallO{\Expn{N_\Po(k_n)}},
\end{align*}
where we again used that $\sqrt{n \log(n)} k_n = \smallO{n}$.
\end{proof}

With this result we can now prove Proposition~\ref{prop:clustering_ast_H_Pois}, which states
\[
	\lim_{n \to \infty} s(k_n)\Exp{\left|c^\ast(k_n; G_n) - c^\ast(k_n; \GPo)\right|} = 0.
\]

\begin{proof}[Proof of Proposition~\ref{prop:clustering_ast_H_Pois}]
First we note that Proposition~\ref{prop:couling_c_H_P}, Proposition~\ref{prop:concentration_local_clustering_P_n}, Proposition~\ref{prop:convergence_average_clustering_P_n} together imply that
\[
	\Exp{c^\ast(k_n; \GPo)} = (1+\smallO{1})s(k_n)
\]
Therefore it suffices to show that
\[
	\Exp{\left|c^\ast(k_n; G_n) - c^\ast(k_n; \GPo)\right|} = \smallO{\Exp{c^\ast(k_n; \GPo)}}.
\]
For this we observe that we are looking at the modified clustering coefficient, where we divide by the expected number of degree $k_n$ vertices. As the expected numbers of degree $k_n$ vertices in $\GPo$ and $G_n$ are asymptotically equivalent (see Lemma~\ref{lem:diff_Nk_hyperbolic_binomial_poisson}), it is therefore sufficient to consider the sum of the clustering coefficients of all vertices of degree $k_n$.
Given again the standard coupling between the binomial and Poisson process (as used in the proof of Lemma~\ref{lem:diff_Nk_hyperbolic_binomial_poisson}), we again denote by $\mathcal{V}_n(k_n)$ the set of degree $k_n$ vertices in $G_n$ and by $\VPo(k_n)$ the set of degree $k_n$ vertices in $\GPo$. If a vertex is contained in both sets, it must have the same degree in both the Poisson and binomial graph, and given the nature of the coupling, the neighbourhoods are therefore the same and hence also their clustering coefficients agree.

The difference of the sum of the clustering coefficients therefore comes from all the clustering coefficients of the symmetric difference $\mathcal{V}_n(k_n) \Delta \VPo(k_n)$. By Lemma~\ref{lem:diff_Nk_hyperbolic_binomial_poisson} the expected number vertices in this set is $\Exp{\left|N_{\Po}(k_n) - N_{\Po}(k_n)\right|} = \smallO{\Exp{N_{\Po}(k_n)}}$. Therefore we have that
\begin{align*}
	\Exp{\left|c^\ast(k_n; G_n) - c^\ast(k_n; \GPo)\right|}
	&\le \frac{\Exp{\left|N_{\Po}(k_n)-N_{\Po}(k_n)\right|}}{((1+\smallO{1})\Exp{N_{\Po}(k_n)}} \Exp{c^\ast(k_n; \GPo)}
	= \smallO{1}\Exp{c^\ast(k_n; \GPo)},
\end{align*}
which finishes the proof.
\end{proof}

Finally we prove Lemma~\ref{lem:clustering_ast_H}, whose statement is
\[
	\Exp{\left|c^\ast(k_n; G_n) - c(k_n; G_n)\right|} = \smallO{s(k_n)}.
\]


\begin{proof}[Proof of Lemma~\ref{lem:clustering_ast_H}]
Let $0 < \delta < 1$ and define the event
\begin{align*}
	A_n &= \left\{\left|N_n(k_n) - \Exp{N_n(k_n)}\right| \le \Exp{N_n(k_n)}^{\frac{1 + \delta}{2}}\right\}.
\end{align*}

Since $N_n(k_n) = \sum_{i = 1}^n \ind{\mathrm{deg}_n(u_i) = k_n}$, with $u_i$ sampled according to~\eqref{eq:def_hyperbolic_point_distribution}, it follows from Lemma~\ref{lem:general_concentration_sum_indicators}, with $c = \Exp{N_{n}(k_n)}^{-\frac{1-\delta}{2}}$, that
\begin{equation}\label{eq:clustering_ast_H_prob_A}
	\Prob{A_n} \ge 1 - \bigO{e^{-\frac{\Exp{N_{n}(k_n)}^\delta}{2}}} = 1 - \bigO{e^{-\frac{n^\delta k_n^{-\delta(2\alpha + 1)}}{2}}},
\end{equation}
where for the last part we used that $\Exp{N_n(k_n)} = \bigT{n k_n^{-(2\alpha + 1)}}$ (cf. Lemma~\ref{lem:diff_Nk_hyperbolic_binomial_poisson}).

On the event $A_n$
\[
	\left|\frac{\Exp{N_{n}(k_n)}}{N_{n}(k_n)} - 1\right| 
	\le \frac{\Exp{N_{n}(k_n)}^{\frac{1 + \delta}{2}}}{\Exp{N_{n}(k_n)}+\Exp{N_{n}(k_n)}^{\frac{1 + \delta}{2}}}
	\le \Exp{N_{n}(k_n)}^{-\frac{1 - \delta}{2}}.
\]

Therefore we have
\begin{align*}
	\Exp{\left|c^\ast(k_n; G_n) - c(k_n; G_n)\right|}
	&\le \Exp{\left|c^\ast(k_n; G_n) - c(k_n; G_n)\right|\ind{A_n}} + \bigO{1 - \Prob{A_n}}\\
	&= \Exp{c^\ast(k_n; G_n)\left|\frac{\Exp{N_{n}(k_n)}}{N_{n}(k_n)} - 1\right|\ind{A_n}}
		+ \bigO{e^{-\frac{n^\delta k_n^{-\delta(2\alpha + 1)}}{2}}}\\
	&\le \Exp{c^\ast(k_n; G_n)}\Exp{N_{n}(k_n)}^{-\frac{1 - \delta}{2}} 
		+ \bigO{e^{-\frac{n^\delta k_n^{-\delta(2\alpha + 1)}}{2}}}.
\end{align*}
The second term is clearly $\smallO{s(k_n)}$. The first term is clearly $\smallO{\Exp{c^\ast(k_n; G_n)}}$ and since Propositions~\ref{prop:clustering_ast_H_Pois}-\ref{prop:convergence_average_clustering_P_n} imply that
\[
	\Exp{c^\ast(k_n; G_n)} = \bigO{s(k_n)}.
\] 
the first term is also $\smallO{s(k_n)}$.
\end{proof}



%\subsection{Degree distribution in the KPKVB graph}






